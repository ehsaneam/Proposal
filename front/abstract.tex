\begin{center}
\textbf{چکیده}
\end{center}
\noindent

رشد سریع کاربردهای نسل چهارم صنعت همچون کارخانه‌های هوشمند نیاز به محاسبات با تأخیر بسیارکم ایجاد می‌کند. محاسبات لبه‌ای چنددسترسی (\lr{\tt{MEC}}) این نیاز را با برون‌سپاری وظایف از دستگاه‌های صنعتی با منابع محدود به سرورهای لبه شبکه برطرف می‌کند. با این حال برون‌سپاری مصالحه‌ای بین پردازش ضعیف محلی مقابل هزینه تأخیر ارتباطی و تداخل سخت‌افزاری پدید می‌آورد. در سرورها اجرای همزمان وظایف موجب رقابت بر سر منابع سخت‌افزاری مشترک، تداخل سخت‌افزاری و افت عملکرد برنامه‌ها می‌شود. مجازی‌سازی (برای مثال کانتینرها) به عنوان راهکاری برای جداسازی و کاهش رقابت پیشنهاد می‌شود. مجازی‌سازی کانتینری هرچند جداسازی توان پردازشی و حافظه را انجام داده است، اما در مواردی مانند حافظه نهان و پهنای باند حافظه، جداسازی امری مشکل است. بنابراین رقابت بر سر منابع سخت‌افزاری هرچند قابل جلوگیری نیست، اما قابل برنامه‌ریزی است.

در مسیر این پژوهش قصد داریم مساله تخصیص منابع پردازش لبه شبکه برای محیط صنعتی هوشمند را مورد حل و بررسی قرار دهیم. تاثیر تداخل سخت‌افزاری بر تصمیم‌گیری بارسپاری، محرک اصلی انتخاب این مسیر است. معماری‌های گوناگون مجازی‌سازی، راهکارهای متمایز کنترل تداخل سخت‌افزاری را پیشنهاد می‌دهند. نیازمندی‌های متنوع در فضای صنعتی از جمله حساسیت به تاخیر، بهره‌وری در مصرف توان و قابلیت اطمینان اهداف متفاوتی برای حل مساله تعریف می‌کند. حل مساله به صورت پویا در گذر زمان، نیازمند تصمیم‌گیری برای مهاجرت برخی وظایف است. هم‌چنین با درک تاثیر تداخل سخت‌افزاری، می‌توان مساله سرمایه‌گذاری اولیه برای هوشمند‌سازی یک فضای صنعتی را مورد حل و بررسی قرار داد.

در پیشنهاد پژوهشی پیش‌رو، شروعی بر مسیر پژوهشی با حل مساله تخصیص منابع و تصمیم‌گیری بارسپاری وظایف در سرورهای‌ لبه شبکه برای نسل نوین صنعت انجام گرفت. این مساله به صورت ایستا مدل شده تا حل یک نمای لحظه‌ای از مساله پویا باشد. نیازمندی حساسیت به تاخیر برای وظایف دستگاه‌ها مورد توجه است. تداخل سخت‌افزاری کانتینرها به صورت کمّی فرموله شده و در مساله بهینه‌سازی موجب مصالحه با قدرت پردازش قوی سرور‌هاست. بر خلاف روش‌های رقیب که تداخل سخت‌افزاری را به صورت کیفی و تحت قواعد هم‌پیوندی مدل می‌کنند، رویکرد پیشنهادی ما عملکرد چشمگیری در نتایج ارائه کرده است. در هر سه شاخصه نرخ حذف درخواست، مولفه‌های آماری تاخیر و افزایش حاشیه امن زمانی تا مهلت اجرای وظیفه، کارایی رویکرد پیشنهادی برتری دارد.

\bigskip
\noindent \textbf{کلیدواژه‌ها}: پردازش لبه شبکه، نسل چهارم صنعت، مدل تداخل سخت‌افزاری، حساس به تاخیر
\newpage
