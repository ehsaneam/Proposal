\begin{center}
\textbf{چکیده}
\end{center}
\noindent

رشد سریع کاربردهای نسل چهارم صنعت، همچون کارخانه‌های هوشمند، ربات‌های خودران و سامانه‌های پایش بلادرنگ، نیاز شدیدی به محاسبات قابل‌اعتماد و کم‌تأخیر ایجاد کرده است. محاسبات لبه‌ای چنددسترسی \lr{\tt{MEC}} این نیاز را با امکان برون‌سپاری وظایف از دستگاه‌های صنعتی با منابع محدود به سرورهای لبه شبکه نزدیک برطرف می‌کند. با این حال، تصمیم‌گیری میان پردازش محلی و برون‌سپاری چالش‌برانگیز است، زیرا برون‌سپاری مستلزم هزینه تأخیر ارتباطی است. علاوه بر آن دومین هزینه برون‌سپاری، مرتبط با محیط مشترک سرورهاست. در سرور تعداد زیادی درخواست به طور همزمان نیازمند پاسخگویی هستند بنابراین رقابت بر سر منابع سخت‌افزاری بین درخواست‌ها شکل می‌گیرد. اجرای وظایف در سرور نیز به‌علت تداخل ناشی از هم‌مکانی برنامه‌ها دچار افت عملکرد می‌شود. این مبادله‌ها به‌ویژه در محیط‌های صنعتی حیاتی هستند، زیرا حتی نقض‌های جزئی مهلت زمانی می‌توانند به شکست تولید یا خطرات ایمنی منجر شوند.

مجازی‌سازی به منظور جداسازی و کاهش رقابت و تداخل پیشنهادهای متنوعی را ارائه می‌کند. یکی از این پیشنهادها کانتینرها هستند. علاوه بر جداسازی، چابکی و سربار کم نیز از ویژگی‌های جذاب کانتینرها برای صاحبان پردازش لبه شبکه است. برنامه‌های مجازی‌سازی شده توسط کانتینر هرچند در استفاده از برخی منابع توسط قابلیت‌های کانتینری جداسازی شده‌اند، اما برخی منابع سخت‌افزاری مشترک هستند و هرگز امکان جداسازی برای آن‌ها وجود ندارد. بنابراین رقابت بر سر منابع سخت‌افزاری مشترک هرچند قابل جلوگیری نیست، اما قابل برنامه‌ریزی است.

این پیشنهاد پژوهشی، یک مدل آگاه از رقابت سخت‌افزاری، با ارزیابی کمّی میزان تداخل، برای برون‌سپاری وظایف در \lr{\tt{MEC}} برای فضای صنعتی هوشمند ارائه می‌دهد. این مدل با توسعه کارهای پیشین، تداخل چندبرنامه‌ای را با تبدیل آمارهای سطح پایین اجرا به عوامل سطح بالای مرتبط با میزان بهره‌برداری از منابع پیشنهاد می‌دهد. مدل تداخل چندبرنامه‌ای توسط هم‌مکانی با تعداد محدودی برنامه محک، محاسبه می‌شود و در نهایت توان تخمین میزان افت کارایی یا میزان کندی برنامه‌های هم‌مکان قابل انجام می‌شود. برخلاف قواعد هم‌پیوندی که با ایده توصیف کیفی تداخل تدابیر لازم را برای جلوگیری از افت کارایی پیش‌بینی می‌کند، رویکرد پیشنهادی ما تخمینی کمّی از تداخل فراهم می‌سازد و امکان ادغام دقیق آن را در چارچوب بهینه‌سازی مساله بهینه‌سازی برای جای‌گذاری وظایف فراهم می‌آورد.

نتایج شبیه‌سازی نشان می‌دهد که الگوریتم پیشنهادی توسط این پژوهش به‌طور چشمگیری از راهکارهای رقیب از جمله رویکرد مبتنی بر هم‌پیوندی بهتر عمل می‌کند؛ به‌گونه‌ای که نرخ حذف درخواست به دلیل تجاوز از مهلت را کاهش داده، مولفه‌های آماری مرتبط با تاخیر را کاهش داده، و حاشیه امن زمانی بهتری برای درخواست‌های حساس به تاخیر فراهم می‌آورد. با پرداختن هم‌زمان به چالش‌های تخصیص منابع محیط \lr{\tt{MEC}} و بحث رقابت سخت‌افزاری، این پژوهش چارچوبی مقاوم و مقیاس‌پذیر برای \lr{\tt{MEC}} در نسل چهارم صنعت فراهم می‌سازد و تضمین می‌کند که کاربردهای حساس به تاخیر با الزامات سخت‌گیرانه عملکرد و قابلیت‌اعتماد را برآورده کند.

\bigskip
\noindent \textbf{کلیدواژه‌ها}: پردازش لبه شبکه، نسل چهارم صنعت، مدل مزاحمت، حساس به تاخیر
\newpage
