
% -------------------------------------------------------
%  English Abstract
% -------------------------------------------------------

\begin{latin}

\begin{center}
\textbf{Abstract}
\end{center}
\baselineskip=.8\baselineskip

The rapid growth of Industry 4.0 applications such as smart factories, creates a strong demand for low-latency computing. Multi-access edge computing (MEC) addresses this need by offloading tasks from resource-constrained industrial devices to edge servers within the network. However, task offloading introduces a trade-off between weak local processing and the communication delay and hardware interference associated with remote execution. On servers, concurrent task execution leads to competition for shared hardware resources, causing interference and performance degradation. Virtualization (e.g., containers) has been proposed as a solution to isolate workloads and reduce resource contention. Although container-based virtualization provides isolation for processing power and memory, complete separation is challenging for shared components such as cache and memory bandwidth. Therefore, while hardware resource contention cannot be entirely eliminated, it can be effectively managed through proper scheduling.

In this research, we aim to address and analyze the problem of resource allocation in edge computing for intelligent industrial environments. The influence of hardware interference on offloading decisions is the primary motivation for this research direction. Different virtualization architectures offer distinct mechanisms for controlling hardware interference. The diverse requirements of industrial environments including latency sensitivity, energy efficiency, and reliability, define multiple objectives for the optimization problem. Solving this problem dynamically over time also requires decision-making regarding task migration. Furthermore, by understanding the effects of hardware interference, one can also investigate the problem of initial investment planning for upgrading to industry 4.0.

In the proposed research, we take the first step toward this direction by addressing the problem of resource allocation and task offloading decisions in edge servers for next-generation industrial systems. The problem is modeled statically to represent a snapshot of the dynamic case. The latency sensitivity of device tasks is taken into account. Hardware interference among containers is quantitatively formulated and incorporated into the optimization model, balancing against the high processing power of servers. Unlike competing approaches that model hardware interference qualitatively using affinity rules, our proposed approach demonstrates significant performance improvements. Across all three key metrics: task drop rate, statistical delay components, and task deadline safety margin, the proposed method outperforms the alternatives.

\bigskip\noindent\textbf{Keywords}: Edge Computing, Industry 4.0, Interference Model, Latency Sensitive

\end{latin}
