
\chapter{مسیر پژوهشی}\label{chap:future}

این فصل مسیرهای آینده‌ی پژوهش برای رساله دکتری را ترسیم می‌کند. توسعه‌ی چارچوب تخمین برای برنامه‌های ناشناخته، گسترش مدل و فرمول‌بندی سامانه از حالت ایستا به پویا با در نظر گرفتن هزینه‌های مهاجرت، بررسی استقرار و امکان‌سنجی اقتصادی \lr{\tt{MEC}} برای نسل چهارم صنعت با تمرکز بر \lr{\tt{CapEx}}، وارد کردن الزامات بیشتر کاربردی مانند قابلیت اطمینان و دسترس‌پذیری، و تحلیل تأثیر ساختارهای مختلف مجازی‌سازی بر رقابت منابع سخت‌افزاری از جمله موضوعات مطرح‌شده هستند. این فصل نقشه‌ی راه تکمیل رساله‌ی کامل را مشخص می‌سازد.

\section{مدل تخمین افت کارایی برنامه ناشناخته}

یکی از دشواری‌های اساسی در مدیریت محیط‌های چندمستاجری \lr{\tt{MEC}} به‌ویژه در معماری کانتینر روی ماشین مجازی، کمّی‌سازی افت کارایی ناشی از رقابت منابع بین برنامه‌های هم‌مکان است. یک رویکرد ساده می‌تواند شامل اندازه‌گیری جامع تداخل برای همه ترکیب‌های ممکن برنامه‌ها باشد؛ با این حال، چنین رویکردی به‌وضوح غیرعملی است، زیرا تعداد ترکیب‌ها به‌صورت نمایی رشد می‌کند. این محدودیت در محیط‌های پویا و صنعتی شدیدتر می‌شود، جایی که برنامه‌های جدید به‌سرعت مستقر می‌شوند و نمایه‌های\LTRfootnote{Profile} تداخل برای کاربار ناشناخته در دسترس نیستند.

برای مقابله با این چالش، ما یک چارچوب تخمین مبتنی بر مدل پیش‌بینی کارایی برنامه‌های ناشناخته تحت رقابت پیشنهاد می‌کنیم. ایده اصلی این است که هرچند رفتار تداخل یک برنامه پیچیده است، اما میزان مصرف منابع آن را می‌توان به‌طور مؤثری با مجموعه‌ای از معیار‌های سطح پایین ثبت کرد. این معیار‌ها شامل دستورالعمل‌ها در هر چرخه (\lr{\tt{IPC}})\LTRfootnote{Instruction per Clock (IPC)}، نرخ موفقیت یا خطای حافظه نهان، مصرف پهنای‌باند حافظه، میزان بهره‌گیری از پهنای‌باند ورودی/خروجی فضای ذخیره‌سازی یا شبکه می‌شوند. با اجرای برنامه ناشناخته به صورت جداگانه و تنها و جمع‌آوری این معیار‌ها، مشخصلات لازم برای کمی‌سازی کارایی برنامه ناشناخته را به دست می‌آوریم\cite{Masouros2021Rusty}.

برای آنکه افت کارایی رفتار برنامه جدید و ناشناخته را در کنار سایر برنامه‌ها مورد سنجش قرار دهیم، تخمین رفتار برنامه ناشناخته را با یک برنامه شناخته شده انجام می‌دهیم. برای انجام این مهم، نیازمند تشکیل یک مجموعه مرجع\LTRfootnote{Dataset} هستیم. تشکیل مجموعه مرجع بر اساس برنامه‌های شناخته شده و معمول مورد استفاده در یک محیط صنعتی صورت می‌گیرد. معیارهای گفته شده برای هر برنامه اندازه‌گیری و ثبت می‌شود. سپس خوشه‌بندی برای جدا کردن دسته‌های مختلف برنامه‌ها بر اساس ویژگی‌های هر کدام از برنامه‌ها انجام می‌شود\cite{Masouros2021Rusty}. سپس ترکیب‌های مختلف برنامه‌های نماینده هر خوشه در معماری کانتینر بر ماشین مجازی در نظر گرفته می‌شود. افت عملکرد ناشی از ترکیب‌های مختلف برنامه‌ها در \lr{\tt{MEC}} نیز مورد اندازه‌گیری قرار می‌گیرد و مجموعه مرجع تشکیل می‌شود.

سپس چارچوب تخمین در دو مرحله عمل می‌کند:

\begin{enumerate}
\item
مرحله نگاشت شباهت به خوشه برنامه‌های شناخته‌شده بر اساس معیارهای اندازه‌گیری شده

\item
تخمین افت کارایی به کمک اندازه‌گیری‌های تجربه شده با برنامه نماینده
\end{enumerate}

سامانه تخمین پیشنهادی چندین مزیت فراهم می‌آورد:

\begin{enumerate}
\item
مقیاس‌پذیری: از رشد نمایی تعداد اندازه‌گیری‌های انواع ترکیب‌ها جلوگیری می‌کند.

\item
عمومیت: از برنامه‌های ناشناخته دلخواه، با نگاشت آن‌ها به نمونه‌های شناخته‌شده پشتیبانی می‌کند.

\item
سازگاری: می‌تواند برنامه‌های جدید را به مجموعه مرجع بیفزاید و دقت پیش‌بینی را غنی‌تر کند.

\item
عملی بودن: بار تجربی را کاهش داده و به‌کارگیری واقعی راهبردهای برون‌سپاری آگاه از تداخل در \lr{\tt{MEC}} را امکان‌پذیر می‌سازد.
\end{enumerate}

با فراهم‌کردن امکان پیش‌بینی تداخل برای کاربارهای ناشناخته، این مدل پایه‌ای برای مدیریت پویای منابع، تصمیم‌های هوشمندانه‌ی برون‌سپاری، و راهبردهای بهینه‌ی جای‌گذاری کانتینر در محیط‌های صنعتی \lr{\tt{MEC}} فراهم می‌سازد.

\section{ارتقای مساله ایستا به پویا}

صورت‌بندی مسئله‌ی جای‌گذاری کانتینر در محیط‌های \lr{\tt{MEC}} یک مدل بهینه‌سازی ایستا را فرض می‌کند. این مدل ایستا می‌تواند یک تصویر لحظه‌ای\LTRfootnote{Snapshot} از شرایط درخواست‌های بارسپاری وظیفه در محیط \lr{\tt{MEC}} باشد. در چنین حالتی، کاربارها تنها یک بار جای‌گذاری می‌شوند و مساله با در نظر گرفتن تداخل منابع با فرض آغاز همزمان کاربارها حل می‌شود. اگرچه این مدل ایستا قابلیت تحلیل‌پذیری دارد و یک سنگ محک\LTRfootnote{Baseline} برای درک مصالحه‌های جای‌گذاری فراهم می‌سازد، اما در عمل با چندین ضعف اساسی روبه‌رو است.

\begin{enumerate}
\item
همانطور که در شکل~\ref{figure:CPU_Comp} دیده می‌شود، زمان‌های اوج مصرف کاربارها ممکن است همزمان نباشند. برنامه‌های مختلف رفتارهای اجرایی گوناگونی نشان می‌دهند؛ برخی ممکن است در بازه‌های کوتاه اوج تقاضای \lr{\tt{CPU}} ایجاد کنند، در حالی‌که برخی دیگر بار سنگین بر حافظه یا ورودی/خروجی را در دوره‌های طولانی‌تری حفظ می‌کنند. جای‌گذاری ایستا قادر به سازگاری با این نوسانات نیست و در نتیجه یا به مصرف زیربهینه منابع یا به بروز تداخل پیش‌بینی‌نشده منجر می‌شود. 

\item
نسبت مصرف اوج به میانگین\LTRfootnote{Peak to average} کاربارها می‌تواند بسیار بالا باشد؛ بنابراین تخصیص منابع بر اساس اوج مصرف بیش از حد محافظه‌کارانه و ناکارآمد خواهد بود، در حالی‌که تخصیص بر اساس میانگین مصرف خطر افت شدید کارایی در زمان‌های اوج را به همراه دارد. 

\item
فرض همزمان بودن آغاز اجرای کاربارها واقع‌بینانه نیست، چرا که وظایف به‌طور پویا و بر اساس رخدادهای ماشینی، قرائت‌های حسگر یا برنامه‌های عملیاتی وارد سامانه می‌شوند.
\end{enumerate}

\vspace{0.5cm}
\begin{figure}[h]
\centering
\includegraphics[width=\textwidth]{sysgauge_process_comparison.pdf}
\caption{تفاوت نسبت قله به میانگین و زمان‌های قله دو کاربار متفاوت.}
\label{figure:CPU_Comp}
\end{figure}
\vspace{0.5cm}

برای غلبه بر این محدودیت‌ها، مسئله‌ی جای‌گذاری باید به یک چارچوب بهینه‌سازی پویا تکامل یابد، جایی که زمان هم مدل‌سازی شود و تصمیم‌های تخصیص منابع با تغییر کاربارها بازنگری گردند. یک رویکرد متداول آن است که زمان به بازه‌های گسسته تقسیم شود\cite{buchaca2020seq2seq} و تخصیص منابع به‌طور جداگانه برای هر بازه انجام گیرد\cite{Masouros2021Rusty}. این روش به سامانه اجازه می‌دهد همراه با تکامل کاربارها جای‌گذاری‌ها را تنظیم کند و تداخل را کاهش دهد.

با این حال، معرفی تقسیم‌بندی زمانی چالش‌های جدیدی را ایجاد می‌کند. به‌طور مشخص، هنگامی که در یک بازه‌ی زمانی تداخل رخ می‌دهد، سامانه با این تصمیم روبه‌رو می‌شود که آیا تداخل را تحمل کند یا کانتینرها را به محلی کم‌ترافیک‌تر مهاجرت دهد. مهاجرت بدون هزینه نیست: این فرایند شامل زمان ازکارافتادگی، انتقال در شبکه، راه‌اندازی مجدد، و احتمال تداخل مجدد در مقصد مهاجرت است\cite{Javadi2017DIAL}. بنابراین، باید یک مدل هزینه در نظر گرفته شود که میان هزینه ناشی از تداخل و هزینه ناشی از مهاجرت تعادل برقرار کند. در برخی موارد، ممکن است ترجیح داده شود که تداخل محدود پذیرفته شود تا اینکه سربار مهاجرت تحمیل گردد، به‌ویژه زمانی که کاربارها کوتاه‌مدت یا گذرا هستند\cite{Anu2019IALM}.

پیچیدگی دیگری که به سامانه تحمیل می‌شود ناشی از ناهماهنگی در مقیاس‌های زمانی است. مقیاس زمانی مناسب برای تخصیص منابع ممکن است با مقیاس زمانی موردنیاز برای تصمیم‌های مهاجرت متفاوت باشد. برای نمونه، تصمیم‌های تخصیص ممکن است لازم باشد هر چند میلی‌ثانیه یک‌بار با توجه به ورود کاربارها بازنگری شوند، در حالی‌که مهاجرت به بازه‌ی زمانی طولانی‌تری نیاز دارد تا بدون ایجاد سربار بیش از حد تکمیل شود. این ناهماهنگی انگیزه‌ای برای استفاده از یک مدل بازه‌ی زمانی چندمقیاسی ایجاد می‌کند، جایی که تخصیص منابع و مهاجرت در مقیاس‌های زمانی متفاوت مدیریت می‌شوند. چنین چارچوبی امکان تنظیمات ریزدانه‌ی تخصیص را فراهم کرده و هم‌زمان قابلیت عملی بودن تصمیم‌های مهاجرت را حفظ می‌کند.

به طور خلاصه ، گذار از بهینه‌سازی ایستا به بهینه‌سازی پویا مدلی غنی‌تر و واقع‌بینانه‌تر از جانمایی کانتینر در سرورهای \lr{\tt{MEC}} ارائه می‌دهد. با در نظر گرفتن هم‌زمان کاربارهای متغیر در زمان، پویایی تداخل، و هزینه‌های مهاجرت، سامانه می‌تواند تضمین‌های قوی‌تری برای کارایی برنامه‌های نسل چهارم صنعت فراهم آورد، جایی که حساسیت به تأخیر، قابلیت اطمینان و دسترسی بالا اهمیت حیاتی دارند. 

\section{بررسی مساله راه‌اندازی و هزینه سرمایه‌گذاری}

تا اینجا، بررسی ما بر تخصیص منابع و مدیریت تداخل در محیط‌های \lr{\tt{MEC}} برای نسل چهارم صنعت متمرکز بوده است. با این حال، فراتر از کارایی عملیاتی، گذار کارخانه‌های سنتی به کارخانه‌های هوشمند پرسش اساسی دیگری را مطرح می‌کند: شرکت‌ها چگونه باید در زیرساخت‌های لازم برای تحقق نسل چهارم صنعت سرمایه‌گذاری کنند؟ این بُعد دامنه پژوهش را از بهینه‌سازی هزینه‌های عملیاتی  (\lr{\tt{OpEx}})\LTRfootnote{Operational Expenditure} به هزینه‌های سرمایه‌ای (\lr{\tt{CapEx}}) گسترش می‌دهد.

وقتی یک کارخانه سنتی قصد دارد به یک کارخانه هوشمند ارتقا یابد، باید ترکیب سرمایه‌گذاری خود را در چندین دسته از فناوری‌های توانمندساز مشخص کند. این دسته‌ها شامل موارد زیر می‌شوند:

\begin{itemize}
\item
\textbf{زیرساخت دسترسی:} استقرار تعداد کافی نقاط دسترسی بی‌سیم و ایستگاه‌های پایه به‌منظور تضمین پوشش پایدار و اتصال کم‌تأخیر.

\item
\textbf{سرورهای \lr{\tt{MEC}} و منابع سخت‌افزاری:} تأمین ظرفیت کافی منابع سخت‌افزاری و شبکه در لبه شبکه به‌منظور پردازش کاربارهای بار‌سپاری‌شده و کاهش تداخل.

\item
\textbf{دستگاه‌های \lr{\tt{IoT}} و حسگرهای هوشمند:} تجهیز ماشین‌آلات و خطوط تولید به واسط‌های دیجیتال که جریان داده برای پایش، کنترل و بهینه‌سازی تولید می‌کنند.

\item
\textbf{توان پردازشی محلی:} حفظ یا ارتقای پردازش محاسباتی در تجهیزات هوشمند کارخانه که می‌توانند در شرایط محدودیت شبکه یا منابع \lr{\tt{MEC}} به‌عنوان جایگزین مورد استفاده قرار گیرند.
\end{itemize}

بینش مهم این است که اثرات تداخل سخت‌افزاری و رقابت بر سر منابع در سرورهای \lr{\tt{MEC}} می‌تواند مستقیماً بر راهبرد سرمایه‌گذاری تأثیر بگذارد. برای مثال، اگر مدل‌سازی تداخل نشان دهد که رقابت منابع باعث کاهش قابل توجه کارایی \lr{\tt{MEC}} برای کاربارهای حساس به تأخیر می‌شود، ممکن است برای یک کارخانه صرفه‌جویی اقتصادی بیشتری داشته باشد که در سخت‌افزار \lr{\tt{MEC}} با ظرفیت بالاتر سرمایه‌گذاری کند یا به‌طور جایگزین، منابع پردازشی محلی را تقویت کند تا کاربارهای حساس به تاخیر را پوشش بدهد. برعکس، اگر کاهش تداخل مؤثر باشد، شرکت می‌تواند از افراط در تأمین سخت‌افزار \lr{\tt{MEC}} خودداری کرده و سرمایه‌گذاری‌ها را به سمت استقرار گسترده‌تر حسگرهای \lr{\tt{IoT}} یا نقاط دسترسی اضافی برای افزایش پوشش هدایت کند.

این دیدگاه انگیزه‌ای برای صورت‌بندی یک مسئله‌ی بهینه‌سازی آگاه به \lr{\tt{CapEx}} ایجاد می‌کند، جایی که هدف تعیین توزیع بهینه‌ی سرمایه‌گذاری در اجزای زیرساخت است، با رعایت محدودیت‌هایی مانند بودجه، ویژگی‌های پیش‌بینی‌شده‌ی کاربارها و تضمین‌های عملکردی (تاخیر، قابلیت اطمینان، دسترسی). چنین صورت‌بندی به شرکت‌ها امکان می‌دهد تا مصالحه‌ها را بررسی کنند: برای مثال، آیا اختصاص بودجه بیشتر به سرورهای \lr{\tt{MEC}} منافع بلندمدت بیشتری نسبت به گسترش زیرساخت حسگری \lr{\tt{IoT}} فراهم می‌کند، یا آیا رویکردهای ترکیبی (ظرفیت متوسط \lr{\tt{MEC}} به همراه پردازش محلی تقویت‌شده) نتایج متعادل‌تری ارائه می‌دهند.

با پیوند دادن مدل‌سازی سامانه آگاه از تداخل با برنامه‌ریزی سرمایه‌گذاری، این جهت پژوهشی یک روش‌شناسی نوآورانه ارائه می‌دهد که کارخانه‌ها، انبارها و سایت‌های صنعتی را در گذار به نسل چهارم صنعت هدایت می‌کند. فراتر از لایه‌ی عملیاتی، این رویکرد به ذی‌نفعان امکان می‌دهد تصمیمات مالی مبتنی بر داده اتخاذ کنند و اطمینان حاصل شود که منابع محدود در جایی سرمایه‌گذاری می‌شوند که بیشترین تأثیر را بر کارایی، مقیاس‌پذیری و پایداری داشته باشند.

\section{بررسی سایر کاربردهای ویژه در صنعت}

در حالی که برنامه‌های حساس به تأخیر تاکنون کانون اصلی توجه این پژوهش بوده‌اند (به دلیل اهمیت پردازش و پاسخ بلادرنگ برای کاربردهای صنعتی که بارسپاری در \lr{\tt{MEC}} انجام می‌دهند) محیط‌های \lr{\tt{MEC}} در نسل چهارم صنعت باید میزبان برنامه‌هایی با الزامات گسترده‌تر و متنوع‌تر نیز باشند. به طور خاص، قابلیت اطمینان، دسترس‌پذیری و بهره‌وری انرژی در کنار تأخیر به عنوان ملاحظات حیاتی پدیدار می‌شوند.

در مدل ارائه شده در این پژوهش، وظایفی که نیازمندی مهلت تأخیر آن‌ها رعایت نمی‌شد حذف می‌شدند\LTRfootnote{Drop} و تخصیص منابع برای سایر وظایف ادامه پیدا می‌کرد. با این حال، در طرح‌های حساس به قابلیت اطمینان، چنین سیاستی ممکن نیست. الزامات قابلیت اطمینان به این معناست که سامانه باید نرخ حذف را مطابق استاندارد برنامه کاربردی رعایت کند. تضمین چنین قابلیتی ممکن است نیازمند راهبردهایی همچون تخصیص منابع بیشتر، افزونگی\LTRfootnote{Redundancy}، یا تخصیص ماشین‌های مجازی اختصاصی باشد تا این کاربارها از افت عملکرد ناشی از برنامه‌های هم‌مکان محافظت شوند.

دسترسی بالای برنامه‌ها بُعد دیگری را معرفی می‌کند. برنامه‌هایی که در \lr{\tt{MEC}} با استفاده از کانتینر فعال هستند، می‌توانند به‌صورت با تقاضا مقیاس یابند. با این حال، هنگامی که از یک کانتینر که برنامه مشخصی را اجرا می‌کند، تنها یک نمونه باشد، و به‌دلیل خرابی سرور، یا دچار مشکل شدن ماشین مجازی یا کانتینر به دلیل تداخل، از دسترس خارج شود، کل خدمت دچار مشکل خواهد شد. برای رفع این مسئله، راهبردهای استقرار باید تکثیر نمونه‌های کانتینر را همچون مقاله \cite{Zhang2020Neighbors} بر ماشین‌های مجازی و سرور‌ها در نظر بگیرند تا دسترس‌پذیری مداوم تضمین شده و خطر قطع سرویس کاهش یابد. اما این کار باعث سربار اضافه شده و منابع محدود \lr{\tt{MEC}} بیشتر استفاده می‌شود.

در نهایت، بهره‌وری انرژی برای دستگاه‌های اینترنت اشیا که اغلب به ظرفیت محدود باتری متکی هستند، غیرقابل‌چشم‌پوشی است. هم محاسبات محلی و هم برون‌سپاری مبتنی بر اتصال به شبکه انرژی مصرف می‌کنند\cite{Kaur2020KEIDS}: اولی از طریق پردازش و دومی از طریق توان انتقال اتصال بی‌سیم. بنابراین سیاست‌های تخصیص منبع برای برنامه‌های حساس به تأخیر باید میان کارایی و مصرف انرژی توازن برقرار کنند تا دستگاه‌های \lr{\tt{IoT}} بتوانند اهداف بلادرنگ خود را بدون تخلیهٔ زودهنگام باتری برآورده سازند. از این منظر، اهداف بهره‌وری انرژی مکمل تضمین‌های تأخیر و قابلیت اطمینان هستند و بهینه‌سازی را پیچیده‌تر می‌سازند، اما آن را بازتاب‌دهنده محدودیت‌های دنیای واقعی می‌کنند.

مجموع این نیازمندی‌ها نشان می‌دهد که تخصیص منبع در \lr{\tt{MEC}} برای نسل چهارم صنعت نمی‌تواند صرفاً بر اساس تأخیر هدایت شود. فرمول‌بندی‌های آینده باید دیدگاه چندهدفه‌ای را اتخاذ کنند که قابلیت اطمینان، دسترس‌پذیری و بهره‌وری انرژی را در کنار تضمین‌های تأخیر ادغام کند، و بدین ترتیب عملکرد سامانه را با الزامات سختگیرانهٔ عملیاتی کارخانه‌های هوشمند و خودکارسازی صنعتی همسو سازد.

\section{بررسی تاثیر معماری‌های مختلف مجازی‌سازی در تداخل سخت‌افزاری}

در این پژوهش، تمرکز اصلی بر مجازی‌سازی کانتینری بوده است، که با هدف استفاده از ویژگی چابکی، سبکی و قابلیت استقرار سریع کانتینرها انتخاب شده است تا برای مصارف حساس به تاخیر نسل چهارم صنعت مناسب باشد. با این حال، ضروری است که تکنیک‌های مجازی‌سازی جایگزین نیز مورد بررسی قرار گیرند تا مشخص شود آیا واقعاً کانتینر با ضعف در جداسازی برای برنامه‌های حساس به تأخیر در محیط‌ صنعتی توام با \lr{\tt{MEC}} گزینه مناسبی است یا خیر.

یک گزینه‌ معماری کانتینر بر ماشین مجازی است. کانتینرها می‌توانند با سرعت بسیار بالا استقرار یافته و مهاجرت کنند، که آن‌ها را سازگار با صنعت می‌کند و از طرفی ماشین‌های مجازی جداسازی مناسبی برای جلوگیری از نشر خطا، مشکل امنیتی و تداخل منابع ایجاد می‌کنند. با این حال، سربار اضافه ناشی از لایه مجازی‌سازی توسط ماشین مجازی موجب افت عملکرد خواهد شد.

فراتر از مدل‌های سنتی، \lr{\tt{Unikernel}} به‌عنوان یک تکنیک نوظهور در مجازی‌سازی مطرح می‌شود. \lr{\tt{Unikernel}} برنامه‌ها را همراه با حداقل مؤلفه‌های موردنیاز سیستم‌عامل بسته‌بندی کرده و مطابق با شکل~\ref{figure:fireVuni} مستقیماً روی هایپروایزر اجرا می‌کنند\cite{firecracker2018}. این مدل وعده‌ی استقرار فوق‌العاده سبک، جداسازی قوی و سطح حمله‌\LTRfootnote{Attack Surface} کمی را می‌دهد، اما از نظر بلوغ زیست‌بوم، ابزارهای سامان‌دهی، و انعطاف‌پذیری نسبت به کانتینرها و ماشین‌های مجازی عقب‌تر است.

\vspace{0.5cm}
\begin{figure}[h]
\centering
\includegraphics[width=0.6\textwidth]{fire.pdf}
\caption{مقایسه ساختار سنتی ماشین مجازی، \lr{\tt{Unikernel}} و \lr{\tt{Firecracker}}\cite{firecracker2018}.}
\label{figure:fireVuni}
\end{figure}
\vspace{0.5cm}

رویکرد قابل توجه دیگر، هایپروایزرهای سبک مانند \lr{\tt{Kata Containers}} یا \lr{\tt{Firecracker}} هستند که هدف آن‌ها پر کردن فاصله میان ماشین‌های مجازی و کانتینرهاست. این فناوری‌ها سرعت و کارایی نزدیک به کانتینرها را فراهم می‌کنند و مشابه آنچه در شکل~\ref{figure:containerVkata} دیده می‌شود، در عین حال جداسازی در سطح ماشین مجازی را حفظ می‌کنند\cite{kata2019}، که می‌تواند مصالحه‌ای جذاب برای استقرار در \lr{\tt{MEC}} باشد، جایی که هم امنیت و هم عملکرد اهمیت بالایی دارند.

\vspace{0.5cm}
\begin{figure}[h]
\centering
\includegraphics[width=\textwidth]{kata.pdf}
\caption{مقایسه ساختار سنتی کانتینر با \lr{\tt{Kata Container}}\cite{kata2019}.}
\label{figure:containerVkata}
\end{figure}
\vspace{0.5cm}

بررسی تأثیر این راهبردهای جایگزین مجازی‌سازی بر تداخل، تأخیر و استفاده از منابع، برای تعیین مؤثرترین معماری مسئله‌ای جذاب خواهد بود. با مقایسه‌ی مجازی‌سازی کانتینری با معماری ترکیبی کانتینر روی ماشین‌مجازی و سایر موارد ذکر شده، پژوهش‌ آتی می‌توانند مصالحه‌های کمی برای استفاده در کاربردهای حساس به تاخیر نسل چهارم صنعت را مشخص کند.