
\chapter{بارسپاری کارآمد لبه شبکه برای نسل چهارم صنعت:}\label{chap:model}

در این فصل نتایج جدید به‌دست‌آمده در پایان‌نامه توضیح داده می‌شود.
در صورت نیاز می‌توان نتایج جدید را در قالب چند فصل ارائه نمود.
همچنین در صورت وجود پیاده‌سازی، بهتر است نتایج پیاده‌سازی را 
در فصل مستقلی پس از این فصل قرار داد.
\section{مزاحمت برنامه‌های کاربردی}
\subsection{جمع‌آوری داده مزاحمت}
\subsubsection{انتخاب مجموعه نرم‌افزارهای معیار}
\subsection{فرایند اندازه‌گیری مزاحمت}
\subsubsection{چیدمان ماشین‌های مجازی}
\section{سیستم مدل پزدارش لبه شبکه در نسل چهارم صنعت}
\subsection{ارسال و دریافت داده برای بارسپاری}
\subsection{مدل مزاحمت}
\section{فرمول‌بندی مساله}
\subsection{تصمیم‌گیری بارسپاری}
\subsection{چیدمان ماشین‌های مجازی}
\subsection{محدودیت منابع پردازشی}
\subsection{مدل مزاحمت و تاخیر پردازش}
\subsection{تابع هدف}
\section{شبیه‌سازی}
\subsection{معیارهای برآورد عملکرد}
\subsection{طرح‌های رقیب}
\section{نتایج}
