
\chapter{مقدمه}\label{chap:intro}

این فصل به معرفی بستر پژوهش، انگیزه‌ها و اهداف رساله می‌پردازد. در آغاز، پیش‌زمینه و انگیزه‌ی بارسپاری محاسبات\LTRfootnote{Computation Offloading} در نسل چهارم صنعت\LTRfootnote{Industry 4.0} و چالش‌های ناشی از رقابت بر سر منابع سخت افزاری ارائه می‌شود. سپس، بیان مسئله صورت‌بندی شده، اهداف پژوهش و پرسش‌های راهنما تعریف می‌گردند و دامنه و دستاوردهای مورد انتظار کار پیشنهادی تشریح می‌شوند. در نهایت، نمای کلی از ساختار رساله ارائه می‌شود تا جریان کلی پژوهش ترسیم گردد.

\section{پیش‌زمینه}

پیشرفت سریع نسل‌ چهارم صنعت، محیط‌های صنعتی سنتی مانند کارخانه‌ها، انبارها، معادن و سکوهای نفتی را به زیست‌بوم‌های\LTRfootnote{Ecosystem} بسیار متصل\LTRfootnote{Highly connected} و خودکار تبدیل کرده است.  در مرکز این تحول، اینترنت اشیاء صنعتی (\lr{\tt{IIoT}})\LTRfootnote{Industrial Internet of Things} قرار دارد؛ جایی که حسگرها\LTRfootnote{Sensors}، عملگرها\LTRfootnote{Actuators}، ربات‌ها و سامانه‌های خودران\LTRfootnote{Autonomous systems} به طور پیوسته داده تولید می‌کنند و نیازمند پردازش بلادرنگ\LTRfootnote{Real-time processing} هستند تا تصمیم‌گیری هوشمند را امکان‌پذیر سازند. بسیاری از این کاربردهای صنعتی (مانند کنترل رباتیک، واقعیت افزوده/مجازی (\lr{\tt{AR/VR}})\LTRfootnote{Augmented/Virtual Reality} برای اپراتورها، نگهداری پیش‌بینانه\LTRfootnote{Predictive maintenance} و وسایل نقلیه خودران) به شدت نسبت به تأخیر حساس‌اند و به تضمین‌های سختگیرانه‌ای در زمینه‌ی تأخیر، قابلیت اطمینان\LTRfootnote{Reliability} و دسترس‌پذیری\LTRfootnote{Availability} نیاز دارند.

اگرچه رایانش ابری\LTRfootnote{Cloud computing} توان محاسباتی چشمگیری را فراهم می‌کند، اما موقعیت جغرافیایی دوردست آن باعث ایجاد تأخیری غیرقابل‌قبول برای بسیاری از وظایف صنعتی می‌شود. در نتیجه، رایانش لبه چنددسترسی (\lr{\tt{MEC}})\LTRfootnote{Multi-access Edge Computing} به‌عنوان یک نگاره\LTRfootnote{Paradigm} آینده دار مطرح شده است که محاسبات را به محل تولید داده نزدیک‌تر می‌کند. با استقرار سرورهای (\lr{\tt{server}}) \lr{\tt{MEC}} در لبه‌ی شبکه\LTRfootnote{Edge network}، سامانه‌های صنعتی می‌توانند پردازش را به منابع مجاور بار‌سپاری\LTRfootnote{Offload} کنند و به‌طور چشمگیری تأخیر را در مقایسه با رایانش ابری کاهش دهند. این امر پاسخ‌گویی بلادرنگ را ممکن می‌سازد که یک الزام کلیدی برای کاربردهای حیاتی\LTRfootnote{Mission-critical applications} نسل چهارم صنعت به شمار می‌رود.

با این حال، سرورهای \lr{\tt{MEC}} در مقایسه با زیرساخت‌های عظیم ابری با محدودیت‌های ذاتی منابع مواجه‌اند. با همزیستی چندین برنامه \LTRfootnote{Application} و خدمت\LTRfootnote{service} (اغلب در قالب کانتینرها (\lr{\tt{Container}}) و ماشین‌های مجازی\LTRfootnote{Virtual Machine} (\lr{\tt{VMs}})) بروز رقابت بر سر منابع امری اجتناب‌ناپذیر است. رقابت بر سر هسته‌های پردازنده، حافظه‌ی نهان\LTRfootnote{cache}، حافظه اصلی\LTRfootnote{Memory}، فضای ذخیره‌سازی\LTRfootnote{Storage} و رابط‌های شبکه\LTRfootnote{Network interfaces}، منجر به تداخل\LTRfootnote{Interference} در کارایی\LTRfootnote{Performance} می‌شود که به‌طور مستقیم بر زمان پاسخ‌دهی به درخواست‌ها اثر می‌گذارد. این تداخل به‌ویژه برای کاربارهای\LTRfootnote{Workloads} صنعتی حساس به تأخیر زیان‌بار است، چرا که حتی تأخیرهای کوچک می‌توانند جریان عملیات را مختل سازند، ایمنی را به خطر اندازند یا موجب خسارات اقتصادی شوند.

فناوری‌های مجازی‌سازی\LTRfootnote{Virtualization} نقشی محوری در محیط‌های \lr{\tt{MEC}} ایفا می‌کنند. ماشین‌های مجازی، جداسازی\LTRfootnote{Isolation} و امنیت قوی فراهم می‌آورند، در حالی که کانتینرها به دلیل سبک‌بار\LTRfootnote{Lightweight} بودن و استقرار سریع برای کاربارهای پویای صنعتی بسیار جذاب‌اند. با این حال، هر دو رویکرد هنگام همزیستی چندین نمونه روی سخت‌افزار مشترک با مشکل رقابت منابع مواجه می‌شوند. افزون بر این، در بسیاری از استقرارهای \lr{\tt{MEC}} هر دو لایه‌ی مجازی‌سازی به‌صورت ترکیبی\LTRfootnote{Hybrid} استفاده می‌شوند؛ به‌طوری که کانتینرها روی ماشین‌های مجازی اجرا می‌گردند تا هم از ویژگی جداسازی ماشین‌های مجازی و هم از چابکی کانتینرها بهره‌مند شوند. این معماری ترکیبی مدیریت منابع و مدل‌سازی تداخل را پیچیده‌تر می‌سازد، به‌ویژه با در نظر گرفتن ماهیت سلسله‌مراتبی رقابت: کانتینرهایی که درون یک ماشین مجازی با هم رقابت می‌کنند، و ماشین‌های مجازی که روی یک سرور فیزیکی \lr{\tt{MEC}} بر سر منابع به رقابت می‌پردازند.

\section{انگیزه پژوهش}

با وجود فعالیت‌ها و پژوهش‌های بسیار در حوزه \lr{\tt{MEC}} برای نسل چهارم صنعت، رویکردهای کنونی در زمینه‌ی بارسپاری محاسبات و جانمایی\LTRfootnote{Placement} کانتینر در محیط‌های صنعتی، اغلب اثر رقابت بر سر منابع را نادیده می‌گیرند. بیشتر روش‌های موجود عمدتاً بر تأخیر ارتباطی و ظرفیت محاسباتی تمرکز دارند، اما در شناسایی افت کارایی ناشی از تداخل میان کاربارهای هم‌مکان\LTRfootnote{Co-located} ناکام می‌مانند. این امر به اتخاذ تصمیم‌های بارسپاری زیر بهینه\LTRfootnote{Sub-optimal} منجر می‌شود: در حالی که بارسپاری به \lr{\tt{MEC}} می‌تواند توان پردازشی بالاتری فراهم کند، اما تأخیرهای ناشی از صف‌بندی و اثرات رقابت منابع ممکن است بر دستاوردها غلبه کرده و در نهایت باعث شود تأخیر اجرای محلی کمتر از اجرای در \lr{\tt{MEC}} باشد.

این مصالحه‌ی\LTRfootnote{Trade-off} پیچیده میان بارسپاری بر \lr{\tt{MEC}} و اجرای محلی، و همچنین میان جدا‌سازی و بهره‌وری منابع، چالش‌های جدیدی را در تضمین کارایی برای کاربردهای صنعتی ایجاد می‌کند. پرداختن به این چالش‌ها انگیزه‌ی اصلی این پژوهش است.

علاوه بر این، کانتینرها ممکن است از تداخل ناشی از کانتینرهای مجاور درون یک ماشین مجازی آسیب ببینند، همان‌طور که رقابت میان ماشین‌های مجازی در سطح ماشین فیزیکی نیز رخ می‌دهد. مدل‌های تحلیل کارایی موجود به ندرت این رقابت را با در نظر گرفتن میزان تاخیر محدود بررسی می‌کنند، و همین موضوع شکاف پژوهشی مهمی در پیش‌بینی دقیق تأخیر و جانمایی بهینه برای طرحهای نسل چهارم صنعت باقی می‌گذارد. بنابراین، مسئله‌ی بنیادی که در این رساله مورد توجه قرار گرفته، چنین است:

چگونه می‌توان یک راهبرد جانمایی کانتینر و بارسپاری محاسبات برای کاربردهای نسل چهارم صنعت در سرورهای \lr{\tt{MEC}} طراحی کرد که رقابت بر سر منابع را در نظر گیرد، تا تأخیر به حداقل رسیده و کارایی برای وظایف حساس به تأخیر تضمین شود؟

\section{اهداف پژوهش}

هدف کلی این رساله توسعه‌ی یک چارچوب آگاه از تداخل\LTRfootnote{Interference aware} برای بارسپاری محاسبات و جانمایی کانتینر در محیط‌های \lr{\tt{MEC}} برای کاربردهای نسل چهارم صنعت است. برای دستیابی به این هدف، این پژوهش به اهداف و پرسش‌های زیر می‌پردازد:

\begin{enumerate}
\item
\textbf{مدل‌سازی رقابت:}
چگونه می‌توان اثر رقابت بر سر منابع را در سرورهای \lr{\tt{MEC}} مدل‌سازی و کمّی‌سازی کرد؟
رقابت بر سر منابع مختلف سخت‌افزاری (پردازنده، حافظه‌ی نهان، حافظه اصلی، ورودی/خروجی (\lr{\tt{I/O}})\LTRfootnote{Input/Ouput} حافظه ذخیره‌سازی، I/O شبکه) چه تأثیری بر کاربردهای حساس به تأخیر دارند؟

\item
\textbf{اندازه‌گیری کارایی:}
چه روش‌های گردآوری داده و پایش می‌توانند برای ثبت دقیق اثرات رقابت در استقرارهای واقعی \lr{\tt{MEC}} به کار گرفته شوند؟
چگونه می‌توان اندازه‌گیری‌های کیفی و کمّی تداخل را در یک مدل یکپارچه‌ی کارا ادغام کرد؟

\item
\textbf{بهینه‌سازی جانمایی کانتینر و بارسپاری:}
چگونه می‌توان مسئله‌ی جانمایی کانتینر را برای کمینه‌سازی تأخیر در شرایط رقابت فرموله کرد؟
چه روش‌های بهینه‌سازی می‌توانند مصالحه میان پردازش محلی، بارسپاری به \lr{\tt{MEC}} و تخصیص منابع آگاه از رقابت را متعادل سازند؟

\item
\textbf{راهبرد مجازی‌سازی با کانتینر:}
مزایا و محدودیت‌های مجازی‌سازی کانتینر در \lr{\tt{MEC}} چیست؟

\end{enumerate}

\section{جایگاه پژوهش}

این رساله بر کاربردهای حساس به تأخیر در نسل چهارم صنعت تمرکز دارد. دامنه‌ی پژوهش به بارسپاری محاسبات و جانمایی کانتینر در سرورهای \lr{\tt{MEC}} محدود می‌شود و گسترش آن به زیرساخت‌های محل توجه اساسی نیست. تأکید اصلی بر مدل‌سازی رقابت منابع، اندازه‌گیری کارایی، و راهبردهای بهینه‌سازی است.
جایگاه اصلی این پژوهش را می‌توان به صورت زیر خلاصه کرد:

\begin{itemize}
\item
\textbf{مدل رقابت بین کانتینر‌ها برای \lr{\tt{MEC}}:}
توسعه‌ی یک چارچوب جامع که تداخل را در سطح کانتینر-به-کانتینر مدل‌سازی می‌کند و پویایی‌های پیچیده‌ی مجازی‌سازی کانتینری در \lr{\tt{MEC}} را پوشش می‌دهد.

\item
\textbf{توصیف تجربی اثرات رقابت:}
اندازه‌گیری‌های گسترده و گردآوری داده‌ها برای توصیف رقابت در میان منابع مختلف (پردازنده، حافظه‌ی نهان، حافظه اصلی، ورودی/خروجی فضای ذخیره‌سازی، ورودی/خروجی شبکه) تحت کاربارهای صنعتی.

\item
\textbf{مسئله‌ی بهینه‌سازی جانمایی کانتینر:}
فرموله‌ کردن یک مسئله‌ی بهینه‌سازی برای جانمایی کانتینر در سرورهای \lr{\tt{MEC}} به‌گونه‌ای که ضمن در نظر گرفتن اثرات رقابت و محدودیت‌های ارتباطی، تأخیر را به حداقل برساند.

\item
\textbf{ارزیابی در طرح‌های متنوع نسل چهارم صنعت:}
اعتبارسنجی رویکرد پیشنهادی از طریق شبیه‌سازی‌ که محیط‌های صنعتی واقعی از جمله کارخانه‌ها، انبارها و عملیات میدانی را بازتاب می‌دهند.

\end{itemize}

ازاین‌رو، این رساله به یک شکاف بسیار مهم در پژوهش‌های \lr{\tt{MEC}} با در نظر گرفتن بارسپاری و جانمایی آگاه از رقابت برای کاربردهای صنعتی حساس به تأخیر می‌پردازد. انتظار می‌رود دستاوردهای این پژوهش هم به پیشبرد مدل‌سازی نظری تداخل در محیط‌های مجازی‌سازی کمک کند و هم راهبردهای استقرار عملی زیرساخت‌های \lr{\tt{MEC}} در نسل چهارم صنعت را بهبود بخشد.

\section{ساختار پایان‌نامه}

باقی‌ فصول پیشنهاد پژوهشی به صورت زیر سازمان‌دهی شده است:
\begin{itemize}
\item
\textbf{\chapref{chap:pre}}
این فصل به معرفی مفاهیم بنیادین مورد نیاز برای رساله می‌پردازد. مبانی نسل چهارم صنعت و نقش کاربردهای حساس به تأخیر در محیط‌های صنعتی را پوشش می‌دهد. همچنین پیش‌زمینه‌ای درباره‌ی \lr{\tt{MEC}}، فناوری‌های مجازی‌سازی (ماشین‌های مجازی و کانتینرها) و پدیده‌ی رقابت بر سر منابع سخت‌افزاری در فضای اشتراکی ارائه می‌دهد. این مقدمات، پایه‌ی علمی لازم برای مسئله‌ی پژوهش را فراهم می‌سازند.

\item
\textbf{\chapref{chap:litr}}
این فصل به بررسی پژوهش‌های موجود مرتبط با \lr{\tt{MEC}} در نسل چهارم صنعت، مجازی‌سازی در \lr{\tt{MEC}} و رقابت بر سر منابع سخت‌افزاری در محیط‌های ابری و لبه‌ شبکه می‌پردازد. همچنین رویکردهای پیشین برای اندازه‌گیری تداخل (چه کیفی و چه کمّی)، تحلیل رقابت در میان منابع سخت‌افزاری، و راهبردهای موجود برای کاهش تداخل را مرور می‌کند. این بحث محدودیت‌های پژوهش‌های کنونی را برجسته ساخته و مسیر پیشنهادی این رساله را توجیه می‌کند.

\item
\textbf{\chapref{chap:model}}
این فصل مدل سامانه‌ی جانمایی کانتینر و بارسپاری در طرح‌های \lr{\tt{MEC}} نسل چهارم صنعت را ارائه می‌دهد. مدل کارا و آگاه از رقابت معرفی شده، مسئله‌ی بهینه‌سازی برای جانمایی کانتینر فرموله می‌گردد و پیکربندی\LTRfootnote{Configuration} شبیه‌سازی توضیح داده می‌شود. نتایج و تحلیل‌ها نیز برای نشان دادن امکان‌پذیری رویکرد پیشنهادی در این فصل قرار دارد.

\item
\textbf{\chapref{chap:future}}
این فصل مسیرهای آینده‌ی پژوهش برای رساله دکتری را ترسیم می‌کند. توسعه‌ی چارچوب تخمین برای برنامه‌های ناشناخته، گسترش مدل و فرمول‌بندی سامانه از حالت ایستا\LTRfootnote{Static} به پویا\LTRfootnote{Dynamic} با در نظر گرفتن هزینه‌های مهاجرت\LTRfootnote{Migration}، بررسی استقرار\LTRfootnote{Deployment} و امکان‌سنجی اقتصادی \lr{\tt{MEC}} برای نسل چهارم صنعت با تمرکز بر هزینه سرمایه‌گذاری (\lr{\tt{CapEx}})\LTRfootnote{Capital Expenditure}، وارد کردن الزامات بیشتر کاربردی مانند قابلیت اطمینان و دسترس‌پذیری، و تحلیل تأثیر ساختارهای مختلف مجازی‌سازی بر رقابت منابع سخت‌افزاری از جمله موضوعات مطرح‌شده هستند. این فصل نقشه‌ی راه تکمیل رساله‌ی کامل را مشخص می‌سازد.

\end{itemize}
