
\chapter{مفاهیم پایه:}\label{chap:pre}

این فصل به معرفی مفاهیم بنیادین مورد نیاز برای رساله می‌پردازد. مبانی نسل چهارم صنعت و نقش کاربردهای حساس به تأخیر در محیط‌های صنعتی را پوشش می‌دهد. همچنین پیش‌زمینه‌ای درباره‌ی \lr{\tt{MEC}}، فناوری‌های مجازی‌سازی (ماشین‌های مجازی و کانتینرها) و پدیده‌ی رقابت بر سر منابع سخت‌افزاری در فضای اشتراکی ارائه می‌دهد. این مقدمات، پایه‌ی علمی لازم برای مسئله‌ی پژوهش را فراهم می‌سازند.

\section{نسل چهارم صنعت}
\subsection{انقلاب صنعتی در گذر زمان}

تاریخچه‌ی انقلاب‌های صنعتی بازتاب‌دهنده‌ی ادغام تدریجی فناوری در سامانه‌های تولید و ساخت است. انقلاب صنعتی نخست (اواخر قرن هجدهم) با ماشینی کردن\LTRfootnote{Mechanization} و نیروی بخار شکل گرفت. انقلاب صنعتی دوم (اواخر قرن نوزدهم و اوایل قرن بیستم) تولید انبوه، استفاده از برق و به‌کارگیری خطوط مونتاژ را معرفی کرد که به‌طور چشمگیری بهره‌وری و مقیاس‌پذیری\LTRfootnote{Scalability} را بهبود بخشید. انقلاب صنعتی سوم (میانه تا اواخر قرن بیستم) دیجیتالی‌ کردن\LTRfootnote{Digitalization}، خودکار کردن\LTRfootnote{Automation} و استفاده از الکترونیک و فناوری اطلاعات\LTRfootnote{Information technology} در فرآیندهای صنعتی را به همراه داشت\cite{Zhou2015Industry4}.

با تکیه بر این تحولات، عصر کنونی یعنی نسل چهارم صنعت نمایانگر انقلاب صنعتی چهارم است (شکل~\ref{figure:4IR_Revs}) که با همگرایی سامانه‌های سایبر-فیزیکی (\lr{\tt{CPS}})\LTRfootnote{Cyber Physical Systems (CPS)}، اینترنت صنعتی اشیا (\lr{\tt{IIoT}})، هوش مصنوعی (\lr{\tt{AI}})\LTRfootnote{Artificial Intelligence (AI)} و اتصال فراگیر\LTRfootnote{Ubiquitous connectivity} مشخص می‌شود\cite{Aceto2019Survey}. این نگاره جدید با تأکید بر تبادل بلادرنگ داده، خودکارسازی هوشمند و سامانه‌های صنعتی خودبهینه‌ساز\LTRfootnote{Self-optimizing} از دوره‌های پیشین متمایز می‌گردد.

\vspace{0.5cm}
\begin{figure}[h]
\centering
\includegraphics[width=\textwidth]{industry_revs.pdf}
\caption{انقلاب‌های صنعتی در گذر زمان\cite{Aceto2019Survey}}
\label{figure:4IR_Revs}
\end{figure}
\vspace{0.5cm}

\subsection{اصول نسل چهارم صنعت}

نسل چهارم صنعت بر مجموعه‌ای از اصول هدایت می‌شود که پایه‌های فناورانه و عملیاتی آن را این گونه تعریف می‌کنند:

\begin{itemize}
\item
\textbf{هم‌کنش‌پذیری\LTRfootnote{Interoperability}:} برقراری ارتباط یکپارچه میان ماشین‌ها، حسگرها، دستگاه‌ها و انسان‌ها از طریق اینترنت صنعتی اشیا و شیوه‌نامه‌های\LTRfootnote{Protocol} استاندارد.

\item
\textbf{مجازی‌سازی:} بازنمایی فرآیندهای فیزیکی در قالب دیجیتال از طریق دوقلوهای دیجیتال(\lr{\tt{DT}})\LTRfootnote{Digital Twin (DT)} و شبیه‌سازی‌ها برای پایش\LTRfootnote{Monitoring} و بهینه‌سازی.

\item
\textbf{غیرمتمرکزسازی\LTRfootnote{Decentralization}:} تصمیم‌گیری خودمختار توسط سامانه‌های سایبر-فیزیکی بدون اتکا صرف به کنترل متمرکز.

\item
\textbf{قابلیت بلادرنگ:} توانایی گردآوری، پردازش و واکنش به جریان‌های داده با حداقل تأخیر.

\item
\textbf{خدمت‌گرایی\LTRfootnote{Service orientation}:} دگرگونی کارکردهای صنعتی به مدل‌های مبتنی بر خدمت، با هدف افزایش انعطاف‌پذیری و مقیاس‌پذیری.

\item
\textbf{پیمانه‌ای بودن\LTRfootnote{Modularity}:} بازپیکربندی انعطاف‌پذیر سامانه‌های تولید برای سازگاری با نیازهای متغیر و تقاضاهای بازار.
\end{itemize}

این اصول در کنار یکدیگر، زیست‌بوم‌های صنعتی هوشمندتر، سازگارتر و تاب‌آورتری را امکان‌پذیر می‌سازند.

\subsection{روندها و نوآوری در نسل چهارم صنعت}

در سال‌های اخیر، سرمایه‌گذاری در فناوری‌های نسل چهارم صنعت رشد سریعی را تجربه کرده است که نشان‌دهنده‌ی افزایش میزان پذیرش و همچنین جایگاه راهبردی آن در صنایع و دولت‌ها است. در سال 2024، ارزش بازار جهانی نسل چهارم صنعت 551.7 میلیارد دلار برآورد شد و انتظار می‌رود تا سال 2030 بیش از دو برابر شده و به حدود 1.6 تریلیون دلار برسد. همچنین نرخ رشد مرکب سالانه (\lr{\tt{CAGR}})\LTRfootnote{Compound Annual Growth Rate (CAGR)} 19.4 درصد را تجربه خواهد کرد\cite{RnM2025GlobMar}. به‌طور مشابه، مطابق شکل ~\ref{figure:4IR_Mar} برای دوره‌ی 2025 تا 2034 اندازه‌ی بازار را با رشد از ۱۹۰ تا ۸۹۲ میلیارد دلار با نرخ رشد مرکب سالانه 18.7 درصد را پیش‌بینی می‌کنند\cite{StatiFacts2025GlobMar}. گزارش‌های دیگر، مانند گزارش \lr{\tt{IMARC}} \cite{IMARC2025GlobInd}، رشد محافظه‌کارانه‌تری را پیش‌بینی می‌کنند و ارزش بازار در سال 2033 را 570.5 میلیارد دلار برآورد می‌نمایند، اما همچنان بازتاب‌دهنده‌ی رشدی دو رقمی و قابل‌توجه  (حدود ۱۴ درصد \lr{\tt{CAGR}} هستند. این ارقام به‌طور کلی بر شتاب اقتصادی پشت سر نسل چهارم صنعت تأکید دارند و اهمیت آن را به‌عنوان یک محور پژوهشی به ‌ویژه برای فناوری‌های کلیدی مانند \lr{\tt{MEC}}، بارسپاری حساس به تأخیر و مجازی‌سازی تأیید می‌کنند.

\vspace{0.5cm}
\begin{figure}[h]
\centering
\includegraphics[width=\textwidth]{industry40market.pdf}
\caption{نمودار پیش‌بینی رشد سرمایه‌گذاری در نسل چهارم صنعت بین سال‌های ۲۰۲۴ تا ۲۰۳۴\cite{StatiFacts2025GlobMar}}
\label{figure:4IR_Mar}
\end{figure}
\vspace{0.5cm}

داده‌های اخیر نشان می‌دهند که نسل چهارم صنعت دیگر یک مفهوم بازار گوشه‌ای\LTRfootnote{Niche concept} نیست، بلکه به یک اولویت راهبردی فراگیر در بسیاری از بخش‌ها تبدیل شده است و فناوری‌های مختلف نقش‌آفرینی ویژه‌ای در این مسیر داشته‌اند (شکل ~\ref{figure:4IR_TechMar}). از نظر سهم فناوری، اینترنت اشیاء \lr{\tt{IoT}} در سال 2023 حدود 20 درصد از بازار را به خود اختصاص داده است که نشان‌دهنده‌ی نقش آن به‌عنوان یک فناوری زیربنایی و توانمندساز کلیدی در نسل چهارم صنعت است\cite{GM2025Rise}. همچنین گزارش \lr{\tt{IoT Analytics}} نشان می‌دهد\cite{IoTAnal2025Rise} که از سال 2011 تاکنون، علاقه به نسل چهارم صنعت رشد چشمگیری داشته است: جستجوی عمومی عبارت \lr{\tt{Industry 4.0}} اکنون حدود 140 برابر بیشتر از زمان آغاز آن است، سرمایه‌گذاری‌های استارتاپی در حوزه‌ی نسل چهارم صنعت بیش از 300 درصد افزایش یافته، و میزان پذیرش راهبردهای مرتبط با آن در شرکت‌ها به‌طور قابل توجهی گسترش یافته است؛ به‌طوری که در یک نظرسنجی سال 2022، 72 درصد از شرکت‌ها گزارش دادند که در حال اجرای ابتکارات کارخانه هوشمند \lr{\tt{Industry 4.0}} یا استفاده گسترده از آن هستند. به‌طور کلی، این روندها هم رشد سریع سرمایه‌گذاری و پذیرش فناوری را برجسته می‌سازند و هم جایگاه ویژه‌ی اینترنت اشیاء را در میان فناوری‌های نسل چهارم صنعت تأیید می‌کنند.

\vspace{0.5cm}
\begin{figure}[h]
\centering
\includegraphics[width=\textwidth]{marketsizebytechnology.pdf}
\caption{نمودار پیش‌بینی تاثیر فناوری‌ها در سرمایه‌گذاری بر نسل چهارم صنعت\cite{GM2025Rise}}
\label{figure:4IR_TechMar}
\end{figure}
\vspace{0.5cm}

چندین نوآوری فناورانه مسیر تکامل نسل چهارم صنعت را شکل می‌دهند\cite{Xu2018Industry4,Tyagi2021Industry4}:

\begin{itemize}
\item
\textbf{هوش مصنوعی و یادگیری ماشین (\lr{\tt{ML}})\LTRfootnote{Machine Learning (ML)}:} به‌کارگیری در نگهداری پیش‌بینانه، تضمین کیفیت و پیش‌بینی تقاضا.

\item
\textbf{تحلیل کلان‌داده\LTRfootnote{Big Data Analytics}:} بهره‌گیری از حجم عظیم داده‌های صنعتی برای استخراج بینش‌های کاربردی.

\item
\textbf{رایانش ابری و \lr{\tt{MEC}}:} پشتیبانی از پردازش توزیع‌شده و کاربردهای صنعتی حساس به تأخیر.

\item
\textbf{شبکه‌‌های نسل پنجم و فراتر از آن\LTRfootnote{5G and beyound}:} فراهم‌سازی ارتباطات فوق‌قابل‌اعتماد با تأخیر بسیار کم (\lr{\tt{URLLC}})\LTRfootnote{Ultra Reliable Low Latency Communication (URLLC)} که برای کاربردهای بلادرنگ حیاتی است.

\item
\textbf{رباتیک و خودکارسازی:} ارتقای بهره‌وری، دقت و ایمنی در عملیات صنعتی.

\item
\textbf{دوقلوهای دیجیتال:} ایجاد نسخه‌های مجازی از دارایی‌های فیزیکی برای پایش، بهینه‌سازی و مدیریت چرخه عمر\LTRfootnote{Lifecycle}.

\item
\textbf{تقویت امنیت سایبری\LTRfootnote{Cybersecurity}:} حفاظت از دستگاه‌ها و شبکه‌های به‌هم‌پیوسته در برابر تهدیدات سایبری.
\end{itemize}

این نوآوری‌ها در کنار هم، محرک دگرگونی به سوی کارخانه‌های هوشمند\LTRfootnote{Smart factories}، زنجیره‌های تأمین متصل و سامانه‌های صنعتی خودگردان هستند.

\subsection{کاربردهای ویژه در صنعت}
\subsubsection{کاربردهای حساس به تاخیر}
بسیاری از کاربردهای نسل چهارم صنعت به‌شدت حساس به تأخیر هستند و نیازمند پردازش و پاسخ بلادرنگ می‌باشند. از جمله می‌توان به کنترل حرکت ربات‌ها، ربات‌های همکار (\lr{\tt{cobots}})\LTRfootnote{Collaborative robots (cobots)}، واقعیت افزوده/مجازی (\lr{\tt{AR/VR}}) برای اپراتورها، و وسایل نقلیه خودران هدایت‌شونده (\lr{\tt{AGVs}})\LTRfootnote{Autonomous Guided Vehicle (AGV)} در انبارها اشاره کرد. این کاربردها به تضمین تأخیر در حد میلی‌ثانیه نیاز دارند و همین امر، \lr{\tt{MEC}} را به یک عامل کلیدی تبدیل می‌کند.

\subsubsection{کاربردهای با اطمینان‌پذیری بالا}
برخی فرآیندهای صنعتی، قابلیت اطمینان را بیش از تأخیر سخت‌گیرانه در اولویت قرار می‌دهند، به‌ویژه در مواردی که عملکرد یکنواخت و بدون خطا اهمیت دارد. نمونه‌هایی از این موارد عبارت‌اند از نگهداری پیش‌بینانه، پایش ایمنی در معادن یا سکوهای نفتی، و سامانه‌های تشخیص خطا. این سامانه‌ها باید با حداقل زمان ازکارافتادگی\LTRfootnote{Downtime} عمل کنند و توان تحمل کاربارهای متغیر و متنوع را بدون افت کارایی داشته باشند.

\subsubsection{کاربردهای نیازمند دسترسی بالا}
در محیط‌های حیاتی مانند سکوهای نفتی، شبکه‌های برق و خطوط تولید پیوسته، دسترسی بالا مهم‌ترین عامل است. سامانه‌ها باید همواره در دسترس و عملیاتی باشند و اغلب به افزونگی\LTRfootnote{Redundancy}، سازوکارهای غلبه بر خرابی \LTRfootnote{Failover} و استقرار توزیع‌شده‌ی  \lr{\tt{MEC}}برای تضمین تداوم خدمت نیاز دارند.

\subsubsection{کاربردهای با مصرف انرژی بهینه}
بهره‌وری انرژی نیز به هدفی مهم در نسل چهارم صنعت بدل شده است. کاربردهایی همچون تولید سبز\LTRfootnote{green manufacturing}، زمان‌بندی\LTRfootnote{Scheduling} ‌آگاه از مصرف انرژی برای ماشین‌های صنعتی، و بهینه‌سازی تدارکات\LTRfootnote{Logistics} در انبارها، نیازمند حداقل‌سازی مصرف انرژی همراه با حفظ کارایی هستند. این کاربردها موجب پذیرش مدل‌های محاسباتی منبع-کارآمد\LTRfootnote{Resource-efficient} شده و مصالحه‌هایی میان محاسبه، ارتباط و مصرف انرژی را ضروری می‌سازند.

\section{پردازش لبه شبکه}

رایانش لبه به‌عنوان یک نگاره کلیدی برای رفع محدودیت‌های زیرساخت‌های ابری متمرکز مطرح شده است، زیرا محاسبات، ذخیره‌سازی و اتصال شبکه را به دستگاه‌های نهایی کاربر نزدیک‌تر می‌کند. در سناریوهای حساس به تأخیر، مانند نسل چهارم صنعت که در آن ماشین‌ها و حسگرها باید داده‌ها را به‌صورت بلادرنگ تبادل کنند، نزدیکی منابع محاسباتی باعث کاهش تأخیر رفت و برگشت و افزایش پاسخ‌دهی خدمات می‌شود. با استقرار سرورها در لبه شبکه در نزدیکی کارخانه‌ها، انبارها و ... ، سازمان‌ها از مزایایی مانند کاهش تأخیر ارتباطی، کاهش ازدحام در زیرساخت شبکه و بهبود قابلیت اطمینان بهره‌مند می‌شوند.

\subsection{معماری محاسبات لبه شبکه با دسترسی چندگانه}

رایانش لبه چنددسترسی \lr{\tt{MEC}} مفهوم رایانش لبه را با ادغام قابلیت‌های محاسباتی به‌طور مستقیم در شبکه دسترسی رادیویی (\lr{\tt{RAN}})\LTRfootnote{Radio Access Network (RAN)} یا در نزدیکی جغرافیایی کاربران گسترش می‌دهد. معماری معمول \lr{\tt{MEC}} برای اینترنت اشیاء شامل سه لایه اصلی است:

\begin{enumerate}
\item
لایه دستگاه، شامل دستگاه‌های اینترنت اشیا، حسگرها، عملگرها و ربات‌های صنعتی

\item
لایه سرور \lr{\tt{MEC}}، که میزبان برنامه‌ها و خدمات مجازی‌سازی‌شده است

\item
لایه شبکه زیرساخت و ابری، که مسئول خدمات متمرکز و تحلیل‌های در مقیاس بزرگ می‌باشد. 
\end{enumerate}

این طراحی لایه‌ای امکان ارتباط با تأخیر فوق‌العاده پایین و مقیاس‌پذیری را فراهم می‌آورد. سرورهای \lr{\tt{MEC}} می‌توانند مانند ابرهای کوچک عمل کنند و با پشتیبانی از مجازی‌سازی و مدیریت هماهنگ\LTRfootnote{Orchestration}، تخصیص پویا منابع و محیط‌های چندمستاجری\LTRfootnote{Multi-tenant} را ارائه دهند که برای کاربردهای نسل چهارم صنعت ضروری است.

\subsection{بارسپاری در لبه شبکه}

در بارسپاری محاسبات، وظایف محاسباتی سنگین از دستگاه‌های محدود از نظر منابع به سرورهای لبه شبکه منتقل می‌شوند. در فضای صنعتی، جایی که سامانه‌های نهفته\LTRfootnote{Embedded systems} یا دستگاه‌های سیار\LTRfootnote{Mobile devices} از نظر انرژی و توان پردازشی محدود هستند، بارسپاری محاسبات این امکان را فراهم می‌کند که وظایفی مانند بینایی ماشین، نگهداری پیش‌بینانه یا کنترل کیفیت مبتنی بر هوش مصنوعی به‌طور کارآمد اجرا شوند. از جمله تصمیمات مهم حوزه بارسپاری به تصمیم‌گیری برای بارسپاری یا پردازش محلی و تصمیم‌گیری برای مهاجرت به سرور‌ی دیگر است.

تصمیم‌گیری در مورد بارسپاری تحت تأثیر عوامل متعددی قرار دارد که معمولاً میان تأخیر، مصرف انرژی، قابلیت اطمینان و هزینه، توازن ایجاد می‌کنند. تأخیر ارتباطی نقش محوری دارد، زیرا تأخیر بیش از حد در انتقال می‌تواند مزایای پردازش سریع‌تر در لبه شبکه را خنثی کند. دسترسی به پهنای باند\LTRfootnote{Bandwidth} و ازدحام شبکه\LTRfootnote{Network Congestion} نیز امکان‌پذیری بارسپاری بلادرنگ را تعیین می‌کنند. از سمت دستگاه، محدودیت‌های باتری و توان پدازشی ضعیف انگیزه‌ای برای بارسپاری به ‌منظور صرفه‌جویی در انرژی و کاهش تاخیر ایجاد می‌کند. از سمت سرور، جداسازی ناکامل منابع سخت‌افزاری میان چند کاربر و تاخیر مخابره می‌تواند به تأخیرهای غیرقابل پیش‌بینی منجر شود و همین امر ضرورت راهبردهای هوشمند در جای‌گذاری خدمت و برنامه کاربردی را تقویت می‌کند. عوامل تصمیم‌گیری دیگر شامل الگوهای جابجایی دستگاه‌ها، توافق‌نامه‌های سطح خدمات (\lr{\tt{SLA}})\LTRfootnote{Service Level Agreement (SLA)} و مهاجرت درخواست‌ها هستند.

\subsection{مجازی‌سازی در محاسبات لبه شبکه}

مجازی‌سازی یکی از ارکان اصلی \lr{\tt{MEC}} است که امکان انتزاع\LTRfootnote{Abstraction} منابع و ارائه خدمات چندکاربره را فراهم می‌کند. این فناوری انعطاف‌پذیری لازم برای میزبانی هم‌زمان چندین برنامه کاربردی را در حالی که امنیت و جداسازی حفظ می‌شود، مهیا می‌سازد. دو رویکرد غالب در این حوزه (ماشین‌های مجازی‌ و کانتینرها)، هر یک مصالحه‌های متفاوتی را از نظر کارایی، جداسازی و سربار\LTRfootnote{Overhead} منابع ارائه می‌کنند.

\subsubsection{ماشین‌های مجازی}

ماشین‌های مجازی بر یک هایپروایزر (\lr{\tt{Hypervisor}}) تکیه دارند تا سخت‌افزار را شبیه‌سازی کرده و محیط‌های جداسازی‌شده‌ای ایجاد کنند که هر یک دارای سیستم‌عامل (\lr{\tt{OS}}) \LTRfootnote{Overhead} مستقل خود هستند. آن‌ها مرزهای امنیتی قوی و جداسازی منابع مؤثری را فراهم می‌سازند و به همین دلیل به‌طور گسترده در استقرارهای ابری و \lr{\tt{MEC}} مورد استفاده قرار می‌گیرند. با این حال، ماشین‌های مجازی معمولاً سربار بیشتری به دلیل تکثیر کامل سیستم‌عامل و زمان راه‌اندازی\LTRfootnote{Startup} طولانی‌تر تحمیل می‌کنند که می‌تواند برای کاربارهای بسیار پویا در نسل چهارم صنعت محدودکننده باشد.

\subsubsection{کانتینر‌ها}

کانتینرها نوعی مجازی‌سازی سبک‌وزن را در سطح سیستم‌عامل پیاده‌سازی می‌کنند؛ به‌طوری که کرنل (\lr{\tt{kernel}}) میزبان\LTRfootnote{Host} را به‌اشتراک می‌گذارند و فرآیندها\LTRfootnote{Processes} را از طریق \lr{\tt{namespace}}ها و \lr{\tt{cgroup}}ها جداسازی می‌نمایند. آن‌ها در مقایسه با ماشین‌های مجازی به منابع کمتری نیاز دارند و می‌توانند ظرف چند میلی‌ثانیه راه‌اندازی شوند، امکانی که استقرار و مقیاس‌پذیری چابک\LTRfootnote{Agile} را فراهم می‌کند. کانتینرها به‌ویژه برای ریزخدمات (\lr{\tt{microservices}}) و کاربردهای پیمانه‌ای، همانند آنچه در سناریوهای نسل چهارم صنعت رایج است، بسیار مناسب‌اند. با این حال، میزان جداسازی آن‌ها نسبت به ماشین‌های مجازی ضعیف‌تر است و تداخل کارایی میان کانتینرهایی که یک هم‌مکان هستند، چالشی شناخته‌شده به شمار می‌رود.

\subsubsection{ماشین مجازی در برابر کانتینر‌ها}

انتخاب میان ماشین‌های مجازی و کانتینرها مستلزم یک مصالحه میان جداسازی و بهره‌وری است. ماشین‌های مجازی در امنیت و پایداری برتری دارند، اما سربار آن‌ها ممکن است برای کاربردهای حساس به تأخیر یا محدود از نظر منابع مناسب نباشد. کانتینرها استقرار سریع و تراکم\LTRfootnote{Density} بالاتر را فراهم می‌کنند، اما هنگامی که چندین کاربر به‌طور مشترک از پردازنده، حافظه، ورودی/خروجی و حافظه پنهان استفاده می‌کنند، ممکن است دچار رقابت بر سر منابع شوند. در محیط‌های نسل چهارم صنعت، که هم امنیت و هم پاسخ‌دهی بلادرنگ اهمیت حیاتی دارند، نه ماشین‌های مجازی و نه کانتینرها به‌تنهایی راه‌حلی کامل ارائه نمی‌دهند.

\subsection{ساختار ترکیبی کانتینر بر ماشین مجازی}

یک رویکرد مهم، مدل ترکیبی است که در آن کانتینرها بر روی ماشین‌های مجازی مستقر می‌شوند. این رویکرد چابکی و بهره‌وری بالای کانتینرها را با جداسازی قدرتمند ماشین‌های مجازی ترکیب می‌کند. در سرورهای \lr{\tt{MEC}}، این معماری امکان کنترل دقیق‌تر بر برنامه‌های کانتینری را فراهم می‌سازد و در عین حال جداسازی کاربران در سطح ماشین مجازی را حفظ می‌کند. با این حال، این مدل اشکال جدیدی از رقابت منابع را نیز به همراه دارد: نه‌تنها کانتینرها درون یک ماشین مجازی با یکدیگر رقابت می‌کنند، بلکه خود ماشین‌های مجازی نیز برای دستیابی به منابع سخت‌افزاری میزبان \lr{\tt{MEC}} در رقابت‌اند. از این‌رو، درک و مدل‌سازی این رقابت چندسطحی برای طراحی راهبردهای کارآمد بارسپاری و جانمایی در کاربردهای حساس به تأخیر نسل چهارم صنعت اهمیت ویژه‌ای دارد.

اسناد شرکت \lr{\tt{vmware}} تأکید می‌کند\cite{vmware_whitepaper} که اگرچه کانتینرها استقرار سبک و بهره‌وری بالای منابع را فراهم می‌کنند، اما فاقد جداسازی قوی هستند زیرا همه کانتینرها از یک کرنل سیستم‌عامل مشترک استفاده می‌کنند. این موضوع مرزهای امنیتی را تضعیف کرده و خطر افت کارایی ناشی از تداخل همسایه‌ مزاحم را افزایش می‌دهد. در مقابل، اجرای برنامه‌ها صرفاً روی ماشین‌های مجازی جداسازی قوی، امنیت و امکانات مدیریتی بالغی ارائه می‌دهد، اما ماشین‌های مجازی سنگین‌تر از کانتینرها بوده و سربار بیشتری تحمیل می‌کنند.

\section{پدیده همسایه مزاحم در محیط مجازی}

در محیط‌های محاسباتی چندکاربره مانند سرورهای \lr{\tt{MEC}}، منابع میان کاربردهای متنوعی که بر روی ماشین‌های مجازی و کانتینرها میزبانی می‌شوند، به‌اشتراک گذاشته می‌شوند. اگرچه این اشتراک‌گذاری بهره‌وری زیرساخت را افزایش می‌دهد، اما خطر رقابت منابع را نیز به همراه دارد؛ جایی که یک کاربار منابع را بیش از حد مصرف کرده و بر کارایی سایر کاربارها تأثیر منفی می‌گذارد. به این پدیده همسایه مزاحم گفته می‌شود. در کاربردهای حساس به تأخیر نسل چهارم صنعت، همسایه‌های مزاحم می‌توانند به نوسانات غیرقابل پیش‌بینی در عملکرد منجر شوند. در نتیجه \lr{\tt{SLA}} و تضمین‌های بلادرنگ را به خطر بیندازند.

\subsection{علل نرم‌افزاری}

تداخل نرم‌افزاری بین برنامه‌ها زمانی رخ می‌دهد که پردازش‌ها برای منابع مشترک کرنل یا سازوکار همگام‌سازی\LTRfootnote{Synchronization} با یکدیگر رقابت کنند. توقف‌ها روی قفل‌های سراسری\LTRfootnote{Global locks}، تداخل در مسیرهای فراخوانی سامانه LTRfootnote{System call}، ساختارهای داده‌ای کرنل و سربار زمان‌بند\LTRfootnote{Scheduler overhead} روی یکدیگر تأثیر بگذارند. این اثرات باعث ایجاد تأخیر و عدم پیش‌بینی در اجرا شده و منجر به کاهش کارایی می‌شوند.

یک مثال عملی در مطالعه موردی تارنوشت \lr{\tt{Sysdig}} آمده است\cite{sysdig2017isolation}، جایی که یک کانتینر مزاحم مطابق شکل~\ref{figure:software_contention} باعث کاهش قابل‌توجه کارایی کانتینر دیگر شد، حتی با وجود آن‌که پردازنده و حافظه سرور تقریبا آزاد است و کمتر از پنج درصد اشغال شده است. مشکل ناشی از رشد حافظه \lr{\tt{dentry}} که مسیر پوشه‌ها را ذخیره‌ می‌کند، ایجاد شده است. حافظه \lr{\tt{dentry}} از یک جدول \lr{\tt{Hash}} استفاده می‌کند و برای دسترسی به محتوای آن نیاز به محاسبه \lr{\tt{Hash}} است. کانتینر مزاحم با تولید مسیرهای متنوع و زیاد پوشه‌ها باعث بزرگ شدن این حافظه و محاسبه طولانی مدت \lr{\tt{Hash}} می‌شود. در نتیجه باعث افزایش تأخیر فراخوانی‌های سامانه‌ای (مانند دستور \lr{\tt{lstat}} مورد استفاده کانتینر اصلی) شد. پس از حذف کانتینر مزاحم، کارایی فوراً بازیابی شد. این موضوع نشان می‌دهد که تداخل می‌تواند ناشی از رفتار نرم‌افزاری باشد. برنامه کاربردی مربوط به کانتینر مزاحم در نسخه‌های بروز‌تر اصلاح و این مشکل حل شد.

\vspace{0.5cm}
\begin{figure}[h]
\centering
\includegraphics[width=\textwidth]{contention.pdf}
\caption{تاثیر کانتینر مزاحم بر زمان پاسخ‌دهی به علت تداخل نرم‌افزاری\cite{sysdig2017isolation}}
\label{figure:software_contention}
\end{figure}
\vspace{0.5cm}

\subsection{علل سخت‌افزاری}

تداخل سخت‌افزاری اصلی‌ترین سازوکاری است که از طریق آن پدیده همسایه مزاحم در محیط‌های \lr{\tt{MEC}} بروز می‌یابد. منابع فیزیکی مشترک (مانند پردازنده‌، حافظه، فضای ذخیره‌سازی و رابط‌های شبکه) زمانی به گلوگاه\LTRfootnote{Bottlenecks} تبدیل می‌شوند که چندین برنامه کاربردی به‌طور هم‌زمان برای دسترسی به آن‌ها رقابت کنند.

\subsubsection{مزاحمت پردازنده}

پردازنده اغلب جدی‌ترین منبع مشترک در سرورهای لبه شبکه محسوب می‌شود. مجازی‌سازی با معرفی سیاست‌های زمان‌بندی \lr{\tt{CPU}}، هسته‌های فیزیکی\LTRfootnote{Physical cores} را میان ماشین‌های مجازی و کانتینرها تسهیم‌\LTRfootnote{Multiplex} می‌کند. با این حال، کاربارهای محاسباتی سنگین می‌توانند هسته‌ها را به‌طور انحصاری اشغال کرده و باعث محروم شدن برنامه‌های کاربری همسایه و افزایش تأخیر در پاسخ آنها شوند. افزون بر این، تداخل تنها به چرخه‌های\LTRfootnote{Physical cores} \lr{\tt{CPU}} محدود نمی‌شود. حافظه پنهان مشترک لایه L2/L3)) ممکن است توسط یک کاربار آلوده شود و نرخ برخورد حافظه پنهان\LTRfootnote{Cache hit rate} کارباری دیگر را کاهش دهند. به همین ترتیب، رقابت بر سر رشته‌های\LTRfootnote{Thread} سخت‌افزاری در معماری‌های چندرشته‌ای همزمان (\lr{\tt{SMT}})\LTRfootnote{Simultaneous Multithreading (SMT)}  نیز به متغیر بودن عملکرد منجر می‌شود.

\subsubsection{مزاحمت حافظه}

رقابت بر سر حافظه از دو سطح ناشی می‌شود: پهنای باند و فضای حافظه مموری (\lr{\tt{memory}}). کاربارهایی با پهنای باند بالای حافظه (مانند تحلیل داده‌ها یا استنتاج\LTRfootnote{Inference} هوش مصنوعی) می‌توانند رابطها را اشباع کنند و دسترسی به حافظه برای برنامه‌های هم‌مکان را با تأخیر مواجه سازند. همچنین اگر جداسازی مناسبی توسط لایه مجازی‌سازی بین برنامه‌های کاربردی انجام نشده باشد، ممکن است کاربارهایی که فضای مموری را بیش از اندازه اشغال می‌کنند باعث کند شدن و انتقال داده از مموری به حافظه \lr{\tt{swap}} در فضای ذخیره‌سازی \lr{\tt{Disk}} می‌شوند.

\subsubsection{مزاحمت فضای ذخیره‌سازی}

فضای ذخیره‌سازی در سرورهای \lr{\tt{MEC}}، دستگاه‌های ورودی/خروجی مانند \lr{\tt{SSD}}\LTRfootnote{Solid State Drives (SSD)}، میان چندین کاربر به‌اشتراک گذاشته می‌شوند. کاربارهای سنگین \lr{\tt{I/O}} می‌توانند باعث انباشت صف در کنترل کننده ذخیره‌سازی شده و در نتیجه، تأخیر \lr{\tt{I/O}} سایر برنامه‌های کاربردی را افزایش دهند. الگوهای دسترسی تصادفی و عملیات خواندن و نوشتن با فرکانس بالا این تداخل را تشدید می‌کنند. در نسل چهارم صنعت، جایی که ثبت و بازیابی سریع داده‌ها ضروری است، رقابت بر سر حافظه ذخیره‌سازی می‌تواند به‌طور چشمگیری پاسخ‌گویی سیستم را کاهش دهد.

\subsubsection{مزاحمت شبکه}

سرورهای \lr{\tt{MEC}} برای پشتیبانی از بارسپاری درخواست‌ها به رابط‌های شبکه مشترک متکی هستند. جریان‌های\LTRfootnote{Flows} شبکه با پهنای باند سنگین\LTRfootnote{Bandwidth intensive} یا انفجاری\LTRfootnote{Bursty} از سوی یک کاربر می‌توانند کارت شبکه (\lr{\tt{NIC}})\LTRfootnote{Network Interface Controller (NIC)} را اشباع کنند و منجر به تأخیر بسته‌ها\LTRfootnote{Packet}، لرزش LTRfootnote{jitter} و از دست رفتن بسته‌ها برای دیگر کاربران شوند. این موضوع به‌ویژه برای درخواست‌های بارسپاری‌شده با مهلت‌های زمانی بلادرنگ زیان‌بار است. علاوه بر این، رقابت منابع می‌تواند به سمت بالادست گسترش یابد و اتصال به شبکه اصلی (\lr{\tt{Backhaul}}) را تحت تأثیر قرار دهد.

\subsection{همسایه مزاحم در مجازی‌سازی}

مجازی‌سازی، لایه‌ای دیگر از پیچیدگی را در رقابت بر سر منابع وارد می‌کند. هم ماشین‌های مجازی و هم کانتینرها از پدیده همسایه مزاحم آسیب می‌بینند، اما سازوکارها و شدت این تأثیر با یکدیگر تفاوت دارند.

\subsubsection{مزاحمت ماشین مجازی}

ماشین‌های مجازی از طریق هایپروایزر جداسازی قدرتمندی فراهم می‌کنند، با این حال چندین ماشین مجازی همچنان برای منابع فیزیکی مشترک در رقابت‌اند. برای نمونه، حافظه پنهان \lr{\tt{CPU}} می‌تواند در صورت رقابت زیاد به نرخ خطای بالای حافظه پنهان منجر شود. افزون بر این، پنهای باند حافظه مموری و دستگاه‌های ورودی/خروجی\LTRfootnote{I/O Devices}، صرف‌نظر از مرزهای\LTRfootnote{Boundaries} هایپروایزر، همچنان در برابر تداخل آسیب‌پذیر باقی می‌مانند.

\subsubsection{مزاحمت کانتینر‌ها}

کانتینرها که کرنل سیستم‌عامل میزبان را به‌اشتراک می‌گذارند، در استفاده از منابع مشترک رقابت تنگاتنگ‌تری را نشان می‌دهند. اگرچه \lr{\tt{cgroup}} و \lr{\tt{namespace}} در کرنل تلاش می‌کنند اشتراک منابع را محدود کنند، اما نمی‌توانند اثرات تداخل را به‌طور کامل جداسازی نمایند. محدودسازی \lr{\tt{CPU}}، اشتراک حافظه پنهان و رقابت بر سر ورودی/خروجی در استقرارهای کانتینری \lr{\tt{MEC}} رایج هستند. کانتینرها علارغم سبک‌تر بودن، تضمین‌های ضعیف‌تری در جداسازی نسبت به ماشین مجازی ارائه می‌دهند.

وقتی برنامه‌های کانتینری روی سرورهای \lr{\tt{MEC}} با سخت‌افزار پیشرفته (مانند پشتیبانی از دسترسی غیریکنواخت به حافظه (\lr{\tt{NUMA}})\LTRfootnote{Non-Uniform Memory Access (NUMA)}) پیاده‌سازی می‌شوند، معماری ترکیبیِ کانتینر روی ماشین مجازی می‌تواند توازن سودمندی از نظر جداسازی، کارایی و بهره‌وری \lr{\tt{CPU}} فراهم کند. بر اساس اسناد شرکت \lr{\tt{VMware}}، اجرای توابع شبکه کانتینری\LTRfootnote{Container Network Function (CNF)} روی ماشین‌های مجازی جداسازی بسیار قوی‌تری نسبت به پیاده‌سازی کانتینرها بدون لایه مجازی‌سازی\LTRfootnote{Bare-metal implementation} ارائه می‌دهد\cite{vmware_whitepaper}. مرزهای ماشین مجازی که توسط هایپروایزر اعمال می‌شوند، لایه‌های محکم امنیت و جداسازی منابع را فراهم کرده و اثرات همسایه مزاحم را که در هنگام اشتراک چندین کانتینر روی یک کرنل فیزیکی شدیدتر است، کاهش می‌دهند. این منبع همچنین نشان می‌دهد که زمان‌بندی \lr{\tt{CPU}} توسط برنامه‌ریز \lr{\tt{vSphere}} که از \lr{\tt{NUMA}}  آگاه است، این قابلیت را دارد که کاربارها را بر اساس نیاز به داده، طوری چیدمان کند که دسترسی‌های این کاربارها بر ماشین‌ مجازی به حافظه محلی با قابلیت \lr{\tt{NUMA}} حداکثری و دسترسی به حافظه غیرمحلی حداقل باشند، و این امر باعث کاهش تأخیر و بهبود کارایی می‌شود.

به‌طور مشابه، پست تارنوشت \lr{\tt{VMware}} گزارش می‌دهد\cite{vmware2019vSphere} که مجموعه کاربارها برنامه‌ریزی شده توسط \lr{\tt{vSphere}} می‌توانند تا هشت درصد کارایی بهتری نسبت به کاربارهای کانتینری در کرنل لینوکس بدون لایه مجازی به دست آورند. این مزیت عمدتاً از آنجا ناشی می‌شود که هایپروایزر \lr{\tt{ESXi}} همراه برنامه‌ریز \lr{\tt{vSphere}} کار زمان‌بندی کانتینرها روی ماشین‌های مجازی را به‌صورت آگاه از \lr{\tt{NUMA}} بهتر انجام می‌دهد و بدین ترتیب دسترسی‌های حافظه غیرمحلی کاهش می‌یابند. در مقابل، کارنتینر در لینوکس بدون لایه مجازی‌سازی ممکن است با کاربارهای مشابه، از تأخیرهای ناشی از دسترسی حافظه غیرمحلی بیشتر رنج ببرد.

\subsection{معیارهای اندازه‌گیری مزاحمت}

برای درک و کاهش مشکل همسایه مزاحم، ثبت و سنجش دقیق تداخل‌ها اهمیت اساسی دارد. رویکردهای سنجش مزاحمت را می‌توان به‌طور کلی به دو دسته کیفی و کمی تقسیم کرد.

\subsubsection{روش‌های کیفی اندازه‌گیری مزاحمت}

این دسته از روش‌ها بر وجود یا عدم وجود مزاحمت تمرکز دارد. همچنین منبع سخت افزاری که توسط یک کاربار تحت فشار قرار می‌گیرد و سطح (کم – متوسط – زیاد) تداخل ایجاد شده بر دیگر کاربارها یا تاثیر گرفته از سایر کاربارها خروجی‌های این روش است. برای مثال هنگامی که اجرای یک کاربار در زمان پیش‌بینی شده انجام نگیرد وجود تداخل را نشان داده و تصمیمات واکنشی از جمله مهاجرت کاربار مورد انتظار است. همچنین برخی برنامه‌های کاربردی که رفتار شناخته شده‌ای در مصرف یک یا چند منبع سخت‌افزاری دارند، در تخصیص منابع به کاربارها مورد توجه قرار گرفته و با یک همسایه مزاحم که رفتار مشابهی دارد در یک مکان قرار نمی‌گیرد. این روش‌ها علارغم آن‌که بینش و تصویر کلی از وجود و ماهیت تداخل می‌دهند،‌ فاقد شاخص‌های دقیق یا قابلیت تعمیم‌پذیری باشند.

\subsubsection{روش‌های کمی اندازه‌گیری مزاحمت}

روش‌های کمی با هدف ثبت میزان تداخل طراحی می‌شوند و از معیارهای اندازه‌گیری مانند چرخه‌های \lr{\tt{CPU}}، نرخ خطای حافظه پنهان، میزان استفاده از پهنای باند حافظه مموری یا تأخیر \lr{\tt{I/O}} بهره می‌گیرند. تحلیل کمی امکان مدل‌سازی ریاضی و ادغام در چارچوب‌های بهینه‌سازی برای مکان‌یابی و تخصیص منابع آگاه از رقابت بر سر منابع را فراهم می‌سازد. برای مثال می‌توان با کمک ابزارهایی مانند \lr{\tt{perf}} یا \lr{\tt{PAPI}} معیارهای یادشده را در دو حالت اجرای برنامه به صورت تنها و اجرا همراه یک برنامه محرک فشارزا\LTRfootnote{Stressor} اندازه‌گیری کرد. سپس با مقایسه نتایج دو اجرا، میزان کاهش کارایی ناشی از مزاحمت بدست می‌آید.

\subsection{روش‌های ترمیم و کنترل مزاحمت}

کاهش اثرات همسایه‌ی مزاحم برای تضمین عملکرد کاربار در سرورهای \lr{\tt{MEC}} بسیار حیاتی است. یکی از تقسیم‌بندی‌های موجود مربوط به زمان انجام اقدامات لازم برای کنترل مزاحمت است. 

روش‌هایی که پس از مشاهده اثرات کاهش کارایی کاربارها، واکنش نشان می‌دهند را روش‌های واکنشی می‌گویند. از جمله اقدامات می‌توان به تصمیم‌گیری مهاجرت اشاره کرد. مهاجرت می‌تواند برای کارباری که موجب کاهش کارایی شده یا برای کارباری که بیشترین آسیب را از مزاحمت دیده انجام گیرد. همچنین تغییر منابع، اولویت و یا زمان‌بندی برای کاربارهای متاثر از مزاحمت از جمله راهکارهای واکنشی هستند. این راهکارها اگرچه ساده هستند، ولی نمی‌توان برای مصارف با حساسیت به تاخیر در نسل چهارم صنعت به آن‌ها تکیه کرد. زیرا اندازه‌گیری و تشخیص مزاحمت، تصمیم‌گیری برای رفع مزاحمت، مهاجرت یا تغییر منابع بسیار زمان‌بر هستند و ممکن است نیازمندی‌های درخواست برآورده نشود.

روش‌های پیش‌گیرانه قبل از وقوع مشکل به علاج آن می‌اندیشد. روش‌های پیشگیرانه می‌تواند بسیار ساده یا پیچیده مدل شود. اگر کاربارهای با برچسب مصرف بالا در یک سرور \lr{\tt{MEC}} قرار نگیرند از مزاحمت پیش‌گیری شده است. اما این مدل ممکن است بهینه یا حتی امکان پذیر نباشد. بنابراین با پیش‌بینی میزان تداخل، کاهش کارایی و تاثیر آن بر زمان اجرای کاربار، می‌توان مکان‌یابی بهینه آگاه به رقابت بر سر منابع را انجام داد. این پیش‌بینی ممکن است بر اساس محاسبات کاهش کارایی در حضور محرک فشارزا در مقایسه با اجرای تنها، صورت گیرد. یا اینکه پیش بینی بر اساس کاهش کارایی برنامه‌های کاربردی در کنار یکدیگر بر اساس نتایج محک\LTRfootnote{Benchmark} انجام بشود.


\begin{itemize}
\item مجموعه‌‌های اعداد: 
$\IN, \IZ, \IZ^+, \IQ, \IR, \IC$
\item مجموعه:
$\set{1, 2, 3}$
\item دنباله‌:
$\seq{1, 2, 3}$
\item سقف و کف:
$\ceil{x}, \floor{x}$
\item اندازه و متمم:
$\card{A}, \setcomp{A}$
\item همنهشتی:
$a \iequiv{n} 1$
یا
$a \equiv 1 \imod{n}$ 
%\item شمردن (عاد کردن):
%$3 \divs n, 2 \ndivs n$
\item ضرب و تقسیم:
$\times, \cdot, \div$
\item سه‌نقطه‌:
$1, 2, \dots, n$
\item کسر و ترکیب:
$\frac{n}{k}, \binom{n}{k}$
\item اجتماع و اشتراک:
$A \cup (B \cap C)$
\item عملگرهای منطقی:
$\neg p \vee (q \wedge r)$

\item پیکان‌ها:
$\rightarrow, \Rightarrow, \leftarrow, \Leftarrow, \leftrightarrow, \Leftrightarrow$
\item عملگرهای مقایسه‌ای:
$\not=, \le, \not\le, \ge, \not\ge$
\item عملگرهای مجموعه‌ای:
$\in, \not\in, \setminus, \subset, \subseteq, \subsetneq, \supset, \supseteq, \supsetneq$

\item جمع و ضرب چندتایی:
$\sum_{i=1}^{n} a_i, \prod_{i=1}^{n} a_i$
\item اجتماع و اشتراک چندتایی:
$\bigcup_{i=1}^{n} A_i, \bigcap_{i=1}^{n} A_i$
\item برخی نمادها:
$\infty, \emptyset, \forall, \exists, \triangle, \angle, \ell, \equiv, \therefore$
\end{itemize}

\bigskip
همچنین می‌توانید با استفاده از نرم‌افزار \lr{Ipe} شکل‌های خود را مستقیما
به صورت \lr{pdf} ایجاد نموده و آن‌ها را با دستورات \lr{\tt{img}} یا  \lr{\tt{centerimg}} 
درون متن درج کنید. برای نمونه، شکل~\ref{figure:directional_graph} را ببینید.


\begin{figure}[ht]
\centerimg{strip}{6.5cm}
\caption{نمونه شکل ایجادشده توسط نرم‌افزار \lr{Ipe}}
\label{figure:directional_graph}
\end{figure}

برای درج جدول می‌توانید با استفاده از دستور  «جدول»
جدول را ایجاد کرده و سپس با دستور  «لوح»  آن را درون متن درج کنید.
برای نمونه جدول~\ref{table:comparative_operators} را ببینید.

\vspace{1.5em}

\begin{table}[ht]
\centering
\caption{عملگرهای مقایسه‌ای}

\begin{tabular}{|c|c|}
\hline 
\bf عملگر & \bf عنوان \\ 
\hline \hline 
\lr{\tt{<}} & کوچک‌تر \\ 
\lr{\tt{>}} & بزرگ‌تر \\
\lr{\tt{==}} &  مساوی \\ 
\lr{\tt{<>}} & نامساوی \\ 
\hline
\end{tabular}

\label{table:comparative_operators}
\end{table}

برای درج الگوریتم می‌توانید از محیط «الگوریتم» استفاده کنید.
یک نمونه در الگوریتم~\ref{algorithm:greedy_vertex_cover} آمده است.

\begin{algorithm}
\label{algorithm:greedy_vertex_cover}
\begin{algorithmic}[1]
\Require گراف $G=(V, E)$
\Ensure یک پوشش رأسی از $G$

\State قرار بده $C = \emptyset$  % \توضیحات{مقداردهی اولیه}
\While{$E$ تهی نیست}
%\If{$|E| > 0$}
%	\state{یک کاری انجام بده}
%\EndIf
\State یال دل‌‌خواه $uv \in E$ را انتخاب کن
\State رأس‌های $u$ و $v$ را به $C$ اضافه کن
\State تمام یال‌های واقع بر $u$ یا $v$ را از $E$ حذف کن
\EndWhile
\State $C$ را برگردان
\end{algorithmic}
\end{algorithm}

برای درج مثال‌ها، قضیه‌ها، لم‌ها و نتیجه‌ها به ترتیب از محیط‌های
«مثال»، «قضیه»، «لم» و «نتیجه» استفاده کنید.
برای درج اثبات قضیه‌ها و لم‌ها  از محیط «اثبات» استفاده کنید.

تعریف‌های داخل متن را با استفاده از دستور «مهم» به صورت \textbf{تیره‌} نشان دهید.
تعریف‌های پایه‌ای‌تر را درون محیط «تعریف» قرار دهید.

\begin{تعریف}[اصل لانه‌کبوتری]
اگر $n+1$ کبوتر یا بیش‌تر درون  $n$ لانه قرار گیرند، آن‌گاه لانه‌ای 
وجود دارد که شامل حداقل دو کبوتر است.
\end{تعریف}
