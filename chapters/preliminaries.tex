
\chapter{مفاهیم اولیه}


دومین فصل پایان‌نامه به طور معمول به معرفی مفاهیمی می‌پردازد که در پایان‌نامه مورد استفاده قرار می‌گیرند.
در این فصل به عنوان یک نمونه، نکات کلی در خصوص نحوه‌ی نگارش پایان‌نامه
و نیز برخی نکات نگارشی به اختصار توضیح داده می‌شوند.

\section{نسل چهارم صنعت}

\subsection{انقلاب صنعتی در گذر زمان}
\subsection{اصول نسل چهارم صنعت}
\subsection{روندها و نوآوری در نسل چهارم صنعت}
\subsection{کاربردهای حساس به تاخیر در صنعت}
\subsubsection{نیازمندی‌های کیفیت خدمت}
\subsubsection{مثالی از آستانه‌ تاخیر در نسل چهارم صنعت}

\section{پردازش لبه شبکه}
\subsection{معماری محاسبات لبه شبکه با دسترسی چندگانه}
\subsubsection{محاسبه در لبه شبکه در برابر محاسبات ابری}
\subsection{بارسپاری در لبه شبکه}
\subsubsection{عوامل تصمیم‌گیری برای بارسپاری}
\subsection{مجازی‌سازی در محاسبات لبه شبکه}
\subsubsection{ماشین‌های مجازی}
\subsubsection{حامل‌ها}
\subsubsection{ماشین مجازی در برابر حامل‌ها}
\subsubsection{ساختار ترکیبی حامل بر ماشین مجازی}

\section{پدیده همسایه مزاحم در محیط مجازی}
\subsection{علل سخت‌افزاری}
\subsubsection{مزاحمت پردازنده}
\subsubsection{مزاحمت حافظه}
\subsubsection{مزاحمت شبکه}
\subsection{مزاحمت با وجود مجازی‌سازی}
\subsubsection{مزاحمت ماشین‌های مجازی}
\subsubsection{مزاحمت حامل‌ها}
\subsection{معیارهای اندازه‌گیری مزاحمت}
\subsection{روش‌های ترمیم و کنترل مزاحمت}
\subsubsection{روش‌های واکنشی}
\subsubsection{روش‌های پیشگیرانه}
\subsubsection{روش‌های کیفی کنترل مزاحمت}
\subsubsection{روش‌های کمی کنترل مزاحمت}


پرونده‌ی اصلی پایان‌نامه در قالب استاندارد\RTLfootnote{
قالب استاندارد پایان‌نامه از نشانی
\href{https://github.com/zarrabi/thesis-template}
{github.com/zarrabi/thesis-template}
قابل دریافت است.}
\lr{\tt{thesis.tex}}  نام دارد.
به ازای هر فصل از پایان‌نامه، یک پرونده در شاخه‌ی \lr{\tt{chapters}} ایجاد نموده
و نام آن را در  \lr{\tt{thesis.tex}} (در قسمت فصل‌ها) درج نمایید.
برای مشاهده‌ی خروجی، پرونده‌ی \lr{\tt{thesis.tex}} را با زی‌لاتک کامپایل کنید.
مشخصات اصلی پایان‌نامه را می‌توانید در پرونده‌ی \lr{\tt{front/info.tex}} ویرایش کنید.

برای درج عبارات ریاضی در داخل متن از \$...\$ و 
برای درج عبارات ریاضی در یک خط مجزا از \$\$...\$\$ یا محیط \lr{equation} 
استفاده کنید. برای مثال عبارت 
$2x + 3y$
در داخل متن و عبارت زیر
\begin{equation}
\sum_{k=0}^{n} \binom{n}{k} = 2^n
\end{equation}
در یک خط مجزا درج شده است. 
دقت کنید که تمامی عبارات ریاضی، از جمله متغیرهای تک‌حرفی مانند $x$ و $y$ باید در محیط ریاضی 
یعنی محصور بین دو علامت \$ باشند. 

برخی علائم ریاضی پرکاربرد در زیر فهرست شده‌اند. 
برای مشاهده‌ی دستور  معادل پرونده‌ی منبع را ببینید.


\begin{itemize}
\item مجموعه‌‌های اعداد: 
$\IN, \IZ, \IZ^+, \IQ, \IR, \IC$
\item مجموعه:
$\set{1, 2, 3}$
\item دنباله‌:
$\seq{1, 2, 3}$
\item سقف و کف:
$\ceil{x}, \floor{x}$
\item اندازه و متمم:
$\card{A}, \setcomp{A}$
\item همنهشتی:
$a \iequiv{n} 1$
یا
$a \equiv 1 \imod{n}$ 
%\item شمردن (عاد کردن):
%$3 \divs n, 2 \ndivs n$
\item ضرب و تقسیم:
$\times, \cdot, \div$
\item سه‌نقطه‌:
$1, 2, \dots, n$
\item کسر و ترکیب:
$\frac{n}{k}, \binom{n}{k}$
\item اجتماع و اشتراک:
$A \cup (B \cap C)$
\item عملگرهای منطقی:
$\neg p \vee (q \wedge r)$

\item پیکان‌ها:
$\rightarrow, \Rightarrow, \leftarrow, \Leftarrow, \leftrightarrow, \Leftrightarrow$
\item عملگرهای مقایسه‌ای:
$\not=, \le, \not\le, \ge, \not\ge$
\item عملگرهای مجموعه‌ای:
$\in, \not\in, \setminus, \subset, \subseteq, \subsetneq, \supset, \supseteq, \supsetneq$

\item جمع و ضرب چندتایی:
$\sum_{i=1}^{n} a_i, \prod_{i=1}^{n} a_i$
\item اجتماع و اشتراک چندتایی:
$\bigcup_{i=1}^{n} A_i, \bigcap_{i=1}^{n} A_i$
\item برخی نمادها:
$\infty, \emptyset, \forall, \exists, \triangle, \angle, \ell, \equiv, \therefore$
\end{itemize}

برای ایجاد یک لیست‌ می‌توانید از محیط‌های «فقرات» و «enumerate» همانند زیر استفاده کنید.

\begin{multicols}{2}
\begin{itemize}
\item مورد اول
\item مورد دوم
\item مورد سوم
\end{itemize}

\begin{enumerate}
\item مورد اول
\item مورد دوم
\item مورد سوم
\end{enumerate}

\end{multicols}

یکی از روش‌های مناسب برای ایجاد شکل استفاده از نرم‌افزار \lr{LaTeX Draw} و سپس
درج خروجی آن به صورت یک فایل \lr{\tt{tex}} درون متن 
با استفاده از دستور  \lr{\tt{fig}} یا \lr{\tt{centerfig}} است.
شکل~\ref{figure:vertex_cover} نمونه‌ای از اشکال ایجادشده با این ابزار را نشان می‌دهد.


\begin{figure}[ht]
\centerfig{cover.tex}{.9}
\caption{یک گراف و پوشش رأسی آن}
\label{figure:vertex_cover}
\end{figure}

\bigskip
همچنین می‌توانید با استفاده از نرم‌افزار \lr{Ipe} شکل‌های خود را مستقیما
به صورت \lr{pdf} ایجاد نموده و آن‌ها را با دستورات \lr{\tt{img}} یا  \lr{\tt{centerimg}} 
درون متن درج کنید. برای نمونه، شکل~\ref{figure:directional_graph} را ببینید.


\begin{figure}[ht]
\centerimg{strip}{6.5cm}
\caption{نمونه شکل ایجادشده توسط نرم‌افزار \lr{Ipe}}
\label{figure:directional_graph}
\end{figure}

برای درج جدول می‌توانید با استفاده از دستور  «جدول»
جدول را ایجاد کرده و سپس با دستور  «لوح»  آن را درون متن درج کنید.
برای نمونه جدول~\ref{table:comparative_operators} را ببینید.

\vspace{1.5em}

\begin{table}[ht]
\centering
\caption{عملگرهای مقایسه‌ای}

\begin{tabular}{|c|c|}
\hline 
\bf عملگر & \bf عنوان \\ 
\hline \hline 
\lr{\tt{<}} & کوچک‌تر \\ 
\lr{\tt{>}} & بزرگ‌تر \\
\lr{\tt{==}} &  مساوی \\ 
\lr{\tt{<>}} & نامساوی \\ 
\hline
\end{tabular}

\label{table:comparative_operators}
\end{table}

برای درج الگوریتم می‌توانید از محیط «الگوریتم» استفاده کنید.
یک نمونه در الگوریتم~\ref{algorithm:greedy_vertex_cover} آمده است.

\begin{algorithm}
\label{algorithm:greedy_vertex_cover}
\begin{algorithmic}[1]
\Require گراف $G=(V, E)$
\Ensure یک پوشش رأسی از $G$

\State قرار بده $C = \emptyset$  % \توضیحات{مقداردهی اولیه}
\While{$E$ تهی نیست}
%\If{$|E| > 0$}
%	\state{یک کاری انجام بده}
%\EndIf
\State یال دل‌‌خواه $uv \in E$ را انتخاب کن
\State رأس‌های $u$ و $v$ را به $C$ اضافه کن
\State تمام یال‌های واقع بر $u$ یا $v$ را از $E$ حذف کن
\EndWhile
\State $C$ را برگردان
\end{algorithmic}
\end{algorithm}

برای درج مثال‌ها، قضیه‌ها، لم‌ها و نتیجه‌ها به ترتیب از محیط‌های
«مثال»، «قضیه»، «لم» و «نتیجه» استفاده کنید.
برای درج اثبات قضیه‌ها و لم‌ها  از محیط «اثبات» استفاده کنید.

تعریف‌های داخل متن را با استفاده از دستور «مهم» به صورت \textbf{تیره‌} نشان دهید.
تعریف‌های پایه‌ای‌تر را درون محیط «تعریف» قرار دهید.

\begin{تعریف}[اصل لانه‌کبوتری]
اگر $n+1$ کبوتر یا بیش‌تر درون  $n$ لانه قرار گیرند، آن‌گاه لانه‌ای 
وجود دارد که شامل حداقل دو کبوتر است.
\end{تعریف}

این فصل حاوی برخی نکات ابتدایی ولی بسیار مهم در نگارش متون فارسی است. 
نکات گردآوری‌شده در این فصل به‌ هیچ‌ وجه کامل نیست، 
ولی دربردارنده‌ی حداقل مواردی است که رعایت آن‌ها در نگارش پایان‌نامه ضروری به نظر می‌رسد.

\begin{enumerate}

\item 
علائم سجاوندی مانند نقطه، ویرگول، دونقطه، نقطه‌ویرگول، علامت سؤال و علامت تعجب % (. ، : ؛ ؟ !) 
بدون فاصله از کلمه‌ی پیشین خود نوشته می‌شوند، ولی بعد از آن‌ها باید یک فاصله‌ قرار گیرد. مانند: من، تو، او.
\item 
علامت‌های پرانتز، آکولاد، کروشه، نقل قول و نظایر آن‌ها بدون فاصله با عبارات داخل خود نوشته می‌شوند، ولی با عبارات اطراف خود یک فاصله دارند. مانند: (این عبارت) یا \{آن عبارت\}.
\item 
دو کلمه‌ی متوالی در یک جمله همواره با یک فاصله از هم جدا می‌شوند، ولی اجزای یک کلمه‌ی مرکب باید با نیم‌فاصله\RTLfootnote{«نیم‌فاصله» فاصله‌‌ای مجازی است که در عین جدا کردن اجزای یک کلمه‌ی مرکب از یک‌دیگر، آن‌ها را نزدیک به هم نگه می‌دارد. معمولاً برای تولید این نوع فاصله در صفحه‌کلید‌های استاندارد از ترکیب Shift+Space استفاده می‌شود.}‌‌
 از هم جدا شوند. مانند: کتاب درس، محبت‌آمیز، دوبخشی.
 \item 
 اجزای فعل‌های مرکب با فاصله از یک‌دیگر نوشته می‌شوند، مانند: تحریر کردن، به سر آمدن.
\end{enumerate}

\begin{enumerate}

\item 
در متون فارسی به جای حروف «ك» و «ي» عربی باید از حروف «ک» و «ی» فارسی استفاده شود. همچنین به جای اعداد عربی مانند ٥ و ٦ باید از اعداد فارسی مانند ۵ و ۶ استفاده نمود. 
برای این کار، توصیه می‌شود صفحه‌کلید‌ فارسی استاندارد را روی سیستم خود فعال کنید.
\item 
عبارات نقل‌قول‌شده یا مؤکد باید درون علامت نقل قولِ «» قرار گیرند، نه ''``. مانند: «کشور ایران».
\item 
کسره‌ی اضافه‌ی بعد از «ه» غیرملفوظ به صورت «ه‌ی» یا «هٔ» نوشته می‌شود. مانند: خانه‌ی علی، دنباله‌ی فیبوناچی.
\\
تبصره‌: اگر «ه» ملفوظ باشد، نیاز به «‌ی» ندارد. مانند: فرمانده دلیر، پادشه خوبان. 

\item 
پایه‌های همزه در کلمات، همیشه «ئـ» است، مانند: مسئله و مسئول، مگر در مواردی که همزه ساکن است که در این ‌صورت باید متناسب با اعراب حرف پیش از خود نوشته شود. مانند: رأس، مؤمن. 

\end{enumerate}

\begin{enumerate}


\item 
علامت استمرار، «می»، توسط نیم‌فاصله از جزء‌ بعدی فعل جدا می‌شود. مانند: می‌رود، می‌توانیم.
\item 
شناسه‌های «ام»، «ای»، «ایم»، «اید» و «اند» توسط نیم‌فاصله، و شناسه‌ی «است» توسط فاصله از کلمه‌ی پیش از خود جدا می‌شوند. مانند: گفته‌ام، گفته‌ای، گفته است.
\item 
علامت جمع «ها» توسط نیم‌فاصله از کلمه‌ی پیش از خود جدا می‌شود. مانند: این‌ها، کتاب‌ها.
\item 
«به» همیشه جدا از کلمه‌ی بعد از خود نوشته می‌شود، مانند: به‌ نام و به آن‌ها، مگر در مواردی که «بـ» صفت یا فعل ساخته است. مانند: بسزا، ببینم.
\item 
«به» همواره با فاصله از کلمه‌ی بعد از خود نوشته می‌شود، مگر در مواردی که «به» جزئی از یک اسم یا صفت مرکب است. مانند: تناظر یک‌به‌یک، سفر به تاریخ. 
%\end{enumerate}
%
%
%
%\begin{enumerate}

%\item 
%اجزای اسم‌ها، صفت‌ها، و قیدهای مرکب توسط نیم‌فاصله از یک‌دیگر جدا می‌شوند. مانند: دانش‌جو، کتاب‌خانه، گفت‌وگو، آن‌گاه، دل‌پذیر.
%
%        تبصره: اجزای منتهی به «هاء ملفوظ» را می‌توان از این قانون مستثنی کرد. مانند: راهنما، رهبر. 

\item 
علامت صفت برتری، «تر»، و علامت صفت برترین، «ترین»، توسط نیم‌فاصله از کلمه‌ی پیش از خود جدا می‌شوند. 
مانند: سنگین‌تر، مهم‌ترین.

        تبصره‌: کلمات «بهتر» و «بهترین» را می‌توان از این قاعده مستثنی نمود. 

\item 
پیشوندها و پسوندهای جامد، چسبیده به کلمه‌ی پیش یا پس از خود نوشته می‌شوند. مانند: همسر، دانشکده، دانشگاه.

        تبصره‌: در مواردی که خواندن کلمه دچار اشکال می‌شود، می‌توان پسوند یا پیشوند را جدا کرد. مانند: هم‌میهن، هم‌ارزی. 

\item 
ضمیرهای متصل چسبیده به کلمه‌ی پیش‌ از خود نوشته می‌شوند. مانند: کتابم، نامت، کلامشان. 

\end{enumerate}

